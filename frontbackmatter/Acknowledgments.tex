\documentclass[../thesis.tex]{subfiles}
\graphicspath{{\subfix{../figs/}}}
\begin{document}

%*******************************************************
% Acknowledgments
%*******************************************************
\pdfbookmark[1]{Acknowledgments}{acknowledgments}

\begingroup
\let\clearpage\relax
\let\cleardoublepage\relax
\let\cleardoublepage\relax
\chapter*{Acknowledgments}

I first want to thank the anonymous referees who have taken on the daunting --- and very little rewarding --- task of reviewing my thesis.
I hope, and it was my intent, to have made it somewhat bearable --- who knows, interesting even?

Next, I would like to express my gratitude to David and José, not only for all the work they have carried out as supervisors of my PhD, but also for how they did it.
I must say that I am not the biggest fan of the idea of working \emph{under the supervision} of someone, and that's why I am grateful I could feel like I was working \emph{with} you to make all the original research presented here come to light.
So I thank you for giving me your trust to pursue my work with such freedom.
And thank you for helping me become a better researcher in general, whether in navigating scientific questions, or the strange world of academia --- and all that it entails.

I also wish to extend my thanks to all the support staff at the IFISC, whether for helping me navigate the administrative meanders of the UIB, in particular thanks to the --- sometimes intimidating, but in the end always helpful --- Marta, or for the technical help from the IT guys, in particular Rubén, for having a solution to every problem, and Antònia, of course, for having done such a great job over so many years to build the huge database we have at the IFISC, without which most of the work presented here would not have been possible.

I would like to thank Martón as well for his warm welcome in the DNDS, and for all the insights he has given me throughout my stay in Vienna and beyond.
And by the way, thanks to all the other people of the department who made my stay more than enjoyable!

% \\

% \begin{es}
% Ahora vienen las personas que, aunque no me hayan ayudado directamente con mi trabajo, han contribuido a ello de manera más personal.
% Sé que os vais a poner celosos si no venís primeros, así que, venga, Marco, Jaime, gracias, más que nada por preocuparos de verdad, por ser tan molestos, en fin, por ser tan buenos amigos.
% Gracias a la gente que me ha acompañado desde el principio: Irene, Maria, (es) Rodri, Javi y Mou (Mou...). Con vosotros no soló he conseguido un C1 en español, pero también un B2 en Irene, y muchos momentos memorables.
% Y en general gracias a la gente de la S07 para todo este ruido que me ha encantado odiar.
% Gracias también a todos para enseñarme los rincones más oscuros de la cultura española, de algunas cosas me gustaría olvidar, pero en general ha sido divertido.
% % lo más recóndito
% No ha bastado para convencerme de que la expatriación puede ser viable para mí, pero la habéis hecho mucho más que tolerable, y no es poco.
% \end{es}
% \\

% \begin{otherlanguage}{french}
% Et c'est sur cette transition des plus habiles que je vais maintenant remercier tous les amis que j'ai à regret dû laisser en France ou ailleurs pour faire ce doctorat.
% % X4, gael
% Savoir que même si on ne se va se revoir qu'après plus de six mois, je pourrai avoir l'impression après 5 minutes de conversation qu'on s'est vus la veille, c'est savoir que j'avais toujours cette zone de confort à laquelle revenir, et c'est inestimable.
% Vous n'avez pas idée du bien fou que ça m'a fait à chaque fois de vous revoir.
% Parce que, bien qu'ayant connu (presque) toute ma vie que le ciel du Nord,
% il me semble tout de même que la misère est surtout bien moins pénible auprès des gens qu'on aime.

% Tiens, en parlant de ça. Merci papa, maman --- ou maman, papa, pas de favoris ici hein.
% que je n'aie rien à envier à tous ces gosses d'ingénieurs, profs, médecins ou cadres en tout genre, car vous, le plus beau métier du monde, vous l'avez réalisé à merveille, et vous le faites toujours.
% \end{otherlanguage}


\endgroup


\end{document}