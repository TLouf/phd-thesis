\documentclass[../thesis.tex]{subfiles}
\graphicspath{{\subfix{../figs/}}}
\begin{document}

%*******************************************************
% Abstract
%*******************************************************
%\renewcommand{\abstractname}{Abstract}
\pdfbookmark[1]{Abstract}{Abstract}
% \addcontentsline{toc}{chapter}{\tocEntry{Abstract}}
\begingroup
\let\cleardoublepage\relax
\let\cleardoublepage\relax

\chapter*{Abstract}
Language has a crucial role in and is greatly influenced by widely different spheres of
society, from simple interpersonal communication to the economy and culture. This is
what makes sociolinguistics, the study of the interactions of language and society, a
complex but decidedly worthwhile endeavour. As a wealth of linguistic data can be
retrieved from online social media, the development of new theoretical models aimed at
uncovering mechanisms underlying sociolinguistic phenomena can be better guided and
tested than ever before. In this thesis, we harness this great potential, and take an
interdisciplinary approach inspired by methods of complexity and data science to
sociolinguistics.

First, we study languages as coherent units that compete with others for speakers, in
order to try to identify the drivers of language death and how coexistence of multiple
languages in an interconnected society might come to be. Crucially, we take into account
the spatial embedding of languages, and first observe it using Twitter data. We find
that two languages can coexist with completely separated communities but also with
communities mixed in space, featuring a large population of bilinguals. We capture this
diversity in coexistence states by introducing a model that considers a potential
cultural attachment for one language that may counteract a globally lower prestige, and
the relative ease to learn a language knowing the other. Both simulations' and
analytical results are used to support our claims.

We then focus on variation within a language to point out the dependence of standard
language use with socio-economic status. Focusing on England, we find
that there is a slight tendency for English Twitter users to make more grammatical
mistakes the lower their income is. This tendency is however very different from one
metropolitan area to another, and actually, it seems to be weaker the more
socio-economic classes mix together. We propose a model that accounts for potentially
different mixing patterns and preferences for a language variety. It reproduces this
effect we observed in toy simulations and more realistic ones.
% TODO: a preciser quand resultats definitifs
We thus find that increased social mixing is crucial to tackle potential social and
economic segregation resulting from this language variation.

Lastly, we leverage the interrelationship between language and culture in a case study
of the United States to define its major cultural regions. From geotagged tweets written
in English, we find the usage hotspots of words found in them to then compute the
principal dimensions of lexical variation. With these, we are able to infer coherent
cultural regions and the topics that define them.

The strength of the results we obtained across quite diverse areas of sociolinguistics
is a mirror of the strength of the approach we took throughout our work. It calls for
further developments of this kind, which are most probably only in their infancy. 

\clearpage

\begin{otherlanguage}{french}
\pdfbookmark[1]{Résumé}{Résumé}
\chapter*{Résumé}
Mon résumé
\end{otherlanguage}

\endgroup

\vfill


\end{document}