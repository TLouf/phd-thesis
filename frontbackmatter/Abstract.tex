\documentclass[../thesis.tex]{subfiles}
\graphicspath{{\subfix{../figs/}}}
\begin{document}

%*******************************************************
% Abstract
%*******************************************************
%\renewcommand{\abstractname}{Abstract}
\pdfbookmark[1]{Abstract}{Abstract}
% \addcontentsline{toc}{chapter}{\tocEntry{Abstract}}
\begingroup
\let\cleardoublepage\relax
\let\cleardoublepage\relax

\chapter*{Abstract}
Language has a crucial role in and is greatly influenced by widely different spheres of
society, from simple interpersonal communication to the economy or culture. This is
what makes sociolinguistics, the study of the interactions of language and society, a
complex but decidedly worthwhile endeavour. As a wealth of linguistic data can be
retrieved from online social media, the development of new theoretical models aimed at
uncovering mechanisms underlying sociolinguistic phenomena can be better guided and
tested than ever before. In this thesis, we harness this great potential, and take an
interdisciplinary approach to sociolinguistics that is inspired by methods of complex
systems and data science.

First, we study languages as coherent units that compete with others for speakers, in
order to try to identify the drivers of language death and how coexistence of multiple
languages in an interconnected society might come to be. Crucially, we take into account
the spatial embedding of languages, and first observe it using Twitter data. We find
that two languages can coexist with completely separated communities but also with
communities mixed in space, featuring a large population of bilinguals. We capture this
diversity of coexistence states by introducing a model that considers a potential
cultural attachment for one language that may counteract a globally lower prestige, as
well as the relative ease to learn a language knowing the other. Both simulations' and
analytic results are used to support our claims.

We then focus on variation within a language to point out a potential dependence of
standard language use with socio-economic status. Focusing on England, we find that
there is a slight tendency for English Twitter users to make more grammatical mistakes
the lower their income is. This tendency is however very different from one metropolitan
area to another, and actually, it seems to be weaker the more socio-economic classes mix
together. We propose a model that accounts for potentially different mixing patterns and
preferences for a language variety. It reproduces this effect we observed in a simple
setting that enables us to analyse it mathematically, but also in more realistic
agent-based simulations.
% TODO: a preciser quand resultats definitifs
We thus find that increased social mixing is crucial to tackle potential social and
economic segregation reflected in this linguistic variation.

Lastly, we leverage the interrelationship between language and culture in a case study
of the United States to define its major cultural regions. From geotagged tweets written
in English, we find the usage hotspots of words found in them to then compute the
principal dimensions of lexical variation. With these, we are able to infer coherent
cultural regions and the topics that define them. This quantitative, automatic analysis
thus provides robust answers to the debate around cultural geography, which has been
historically marked by differing definitions of relevant cultural factors.

The strength of the results we obtained across quite diverse areas of sociolinguistics
is a mirror of the strength of the approach we took throughout our work, that relies on
computational tools, large datasets and simple mathematical models. It calls for further
developments of this kind, which are most probably only in their infancy. 

\clearpage

\begin{otherlanguage}{french}
\pdfbookmark[1]{Résumé}{Résumé}
\chapter*{Résumé}
Le langage a un rôle central et subit de fortes influences venant de sphères très
diverses de la société, qui vont de la simple communication entre individus à l'économie
ou la culture. C'est cela qui rend la sociolinguistique, c'est-à-dire l'étude des
interactions entre le langage et la société, une entreprise à la fois complexe et
incontestablement digne d'intérêt. Alors qu'une quantité inédite de données peuvent être
extraites des réseaux sociaux, le développement de nouveaux modèles théoriques qui
visent à identifier les mécanismes sous-jacents aux phénomènes sociolinguistiques peut
être mieux guidé et mis à l'épreuve que jamais auparavant. Dans cette thèse, nous tirons
parti de ce potentiel, et prenons une approche interdisciplinaire à la sociolinguistique
inspirée par des méthodes de science des systèmes complexes et de la science des
données.

En premier lieu, nous étudions les langues comme des unités cohérentes qui sont en
compétition les unes avec les autres afin d'essayer d'identifier les principaux facteurs
qui mènent à la mort d'une langue, et ce qui pourrait rendre possible la coexistence de
multiples langues dans une société interconnectée. Un aspect crucial que nous prenons en
compte est l'ancrage géographique des langues, que nous observons à travers des données
de Twitter. Ces observations nous montrent que deux langues peuvent coexister à travers
deux communautés complètement séparées, mais aussi lorsque ces dernières cohabitent,
avec une population considérable de bilingues. Afin de saisir cette diversité d'états de
coexistence, nous introduisons un modèle qui considère la possibilité d'un attachement
culturel à l'une des langues qui pourrait contrebalancer un prestige globalement
inférieur, ainsi que la facilité relative d'apprentissage d'une langue sachant parler
l'autre. Nos résultats découlant à la fois d'analyse mathématique et de simulations
numériques viennent appuyer nos thèses.

Par la suite, nous nous focalisons sur les variations intrinsèques à une langue afin
d'identifier une potentielle dépendance entre le respect des normes standard d'une
langue et le status socio-économique des individus. Nous concentrons notre analyse sur
l'Angleterre et identifions une légère tendance pour les utilisateurs Anglais de Twitter
de commettre plus d'erreurs grammaticales s'ils ont un revenu plus bas. Cette tendance
est en revanche très différente d'une métropole à l'autre, et, de fait, notre analyse
indique qu'elle soit plus faible quand différentes classes sociales se mélangent plus.
Nous proposons alors un modèle qui prend en compte un plus ou moins grand brassage
social et de potentielles préférences de certaines classes pour une variété
linguistique. Il reproduit l'effet que nous avons observé à la fois dans des
configurations très simples qui nous permettent de l'analyser mathématiquement, mais
également dans des simulations plus réalistes à base d'agents. Nos résultats indiquent
donc que plus de brassage social est crucial si l'on souhaite contrecarrer de
potentielles ségrégations économiques et sociales qui se reflètent dans cette variation
linguistique.

La dernière étude que nous présentons utilise le caractère indissociable de la relation
entre langage et culture dans une étude de cas des États-Unis afin de définir ses
principales régions culturelles. À partir de tweets géolocalisés écrits en anglais, nous
cartographions les zones où certains mots sont utilisés plus ou moins que de coutume
pour ensuite déterminer les principales dimensions de variation lexicale dans le pays.
Nous pouvons alors déduire de celles-ci les principales régions culturelles et les
sujets qui les définissent. Cette approche quantitative et automatique fournit ainsi des
réponses robustes au débat qui entoure la géographie culturelle, historiquement marquée
par des manières différentes de définir les dimensions culturelles pertinentes.

La force des résultats que nous avons obtenus à travers des domaines assez variés de la
sociolinguistique n'est que le miroir des forces de l'approche que nous avons adoptée
tout au long de cette thèse, qui repose sur des outils computationnels, de grands
ensembles de données et des modèles mathématiques simples. Cela invite donc à de plus
amples études de ce type, qui ne sont probablement qu'à leur genèse.
\end{otherlanguage}


\clearpage

\begin{es}
\pdfbookmark[1]{Resumen}{Resumen}
\chapter*{Resumen}
El lenguaje tiene un papel crucial y está bajo una influencia muy fuerte por parte de
esferas de la sociedad muy diversas, que van de la simple comunicación interpersonal a
la economía o la cultura. Eso es lo que hace la sociolingüística, que consiste en el
estudio de las interacciones entre el lenguaje y la sociedad, una empresa a la vez muy
compleja y que vale decididamente la pena llevar a cabo. Ya que se puede extraer
muchísimos datos lingüísticos de las redes sociales, el desarrollo de nuevos modelos
teóricos que intentan revelar los mecanismos detrás de los fenómenos sociolingüísticos
puede estar mejor guiado y testado que nunca. En esta tesis, aprovechamos de este gran
potencial, y tomamos un enfoque interdisciplinario inspirado de métodos de los campos de
los sistemas complejos y de la ciencia de datos.

En primer lugar, estudiamos las lenguas como unidades coherentes que compiten con otras
por hablantes, para tratar de identificar las causas de la muerte de las lenguas, y de
una posible coexistencia de varias lenguas en una sociedad interconectada. Crucialmente,
tenemos en cuenta la incrustación espacial de las lenguas, y la observamos por primera
vez utilizando datos de Twitter. Descubrimos que dos lenguas pueden coexistir con
comunidades completamente separadas, pero también con comunidades mezcladas en el
espacio, con una gran población de bilingües. Captamos esta diversidad de estados de
coexistencia introduciendo un modelo que tiene en cuenta un posible apego cultural por
una lengua que puede contrarrestar un prestigio globalmente inferior, así como la
relativa facilidad para aprender una lengua conociendo la otra. Tanto las simulaciones
como los resultados analíticos se utilizan para apoyar nuestras afirmaciones. 

A continuación nos centramos en la variación dentro de una lengua para señalar una
posible dependencia de uso de la lengua estándar con el estatus socioeconómico.
Centrándonos en Inglaterra, observamos que hay una ligera tendencia a que los usuarios
ingleses de Twitter cometan más errores gramaticales cuanto más bajos son sus ingresos.
Sin embargo, esta tendencia es muy diferente entre un área metropolitana y otra, y, de
hecho, parece ser más débil cuanto más se mezclan las clases socioeconómicas. Proponemos
entonces un modelo que tiene en cuenta las posibles diferencias en los patrones de
mezcla y la preferencia por una variedad lingüística. Reproduce este efecto que
observamos en un entorno sencillo que nos permite analizarlo matemáticamente, pero
también en simulaciones más realistas basadas en agentes. Así pues, consideramos que el
aumento de la mezcla social es crucial para hacer frente a la posible segregación social
y económica reflejada en este fenómeno lingüístico.

Por último, aprovechamos la interrelación entre lengua y cultura en un estudio de caso
de Estados Unidos para definir sus principales regiones culturales. A partir de tuits
geoetiquetados escritos en inglés, hallamos los hotspots de uso de las palabras
encontradas en ellos para luego computar las principales dimensiones de la variación
léxica. Con ellas, podemos inferir regiones culturales coherentes y los temas que las
definen. Este análisis cuantitativo y automático aporta respuestas sólidas al debate en
torno a la geografía cultural, que se ha caracterizado históricamente por las distintas
definiciones de los factores culturales relevantes.

La solidez de los resultados que hemos obtenido en ámbitos bastante diversos de la
sociolingüística refleja la solidez del enfoque que hemos adoptado a lo largo de nuestro
trabajo, que se basa en herramientas informáticas, grandes conjuntos de datos y modelos
matemáticos sencillos. Ello exige nuevos avances de este tipo, que con toda probabilidad
todavía se encuentran en su infancia.
\end{es}



\clearpage

\begin{otherlanguage}{catalan}
\pdfbookmark[1]{Resum}{Resum}
\chapter*{Resum}
La llengua té un paper crucial i està molt influenciada per esferes molt diferents de la
societat, des de la simple comunicació interpersonal a l'economia o la cultura. Això és
el que fa de la sociolingüística, l'estudi de les interaccions del llenguatge i la
societat, un esforç complex però decididament valuós. Com que es pot recuperar una gran
quantitat de dades lingüístiques de les xarxes socials en línia, el desenvolupament de
nous models teòrics destinats a descobrir mecanismes subjacents als fenòmens
sociolingüístics pot ser millor guiat i provat que mai. En aquesta tesi, aprofitem
aquest gran potencial i adoptem un enfocament interdisciplinari de la sociolingüística
que s'inspira en mètodes de sistemes complexos i ciència de dades.

En primer lloc, estudiem les llengües com a unitats coherents que competeixen amb altres
per parlants, per tal d'intentar identificar els conductors de la mort de la llengua i
com podria arribar a ser la coexistència de diverses llengües en una societat
interconnectada. Curiosament, tenim en compte la incrustació espacial de llengües, i
primer observem-la utilitzant dades de Twitter. Ens trobem que dues llengües poden
coexistir amb comunitats completament separades però també amb comunitats barrejades en
l'espai, amb una gran població de bilingües. Captem aquesta diversitat d'estats de
coexistència introduint un model que considera una possible afecció cultural per una
llengua que pot contrarestar un prestigi globalment inferior, així com la relativa
facilitat per aprendre una llengua que conegui l'altra. Tant les simulacions com els
resultats analítics s'utilitzen per donar suport a les nostres afirmacions.

Després ens centrem en la variació dins d'una llengua per assenyalar una dependència
potencial de l'ús estàndard de la llengua amb estat socioeconòmic. Centrant-nos en
Anglaterra, trobem que hi ha una lleugera tendència dels usuaris anglesos de Twitter a
cometre errors més gramaticals com més baixos siguin els seus ingressos. No obstant
això, aquesta tendència és molt diferent d'una àrea metropolitana a una altra, i en
realitat, sembla ser més feble les classes més socioeconòmiques es barregen entre si.
Proposem un model que tingui en compte patrons de barreja i preferències potencialment
diferents per a una varietat lingüística. Reprodueix aquest efecte que hem observat en
un entorn senzill que ens permet analitzar-lo matemàticament, però també en simulacions
basades en agents més realistes.

Finalment, aprofitem la interrelació entre llengua i cultura en un estudi de casos dels
Estats Units per a definir les seves principals regions culturals. A partir de tuits
geoetiquetats escrits en anglès, trobem els punts d'ús de les paraules que es troben en
ells per després calcular les principals dimensions de la variació lèxica. Amb aquests,
podem inferir regions culturals coherents i els temes que les defineixen. Aquesta
anàlisi quantitativa i automàtica, per tant, proporciona respostes sòlides al debat al
voltant de la geografia cultural, que ha estat històricament marcada per diferents
definicions de factors culturals rellevants.

La força dels resultats obtinguts a través de diverses àrees de sociolingüística és un
mirall de la força de l'enfocament que vam prendre durant el nostre treball, que es basa
en eines computacionals, grans conjunts de dades i models matemàtics senzills. Exigeix
més desenvolupaments d'aquest tipus, que molt probablement només estan en la seva
infància.
\end{otherlanguage}

\endgroup

\vfill


\end{document}