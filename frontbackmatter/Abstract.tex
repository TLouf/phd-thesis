\documentclass[../thesis.tex]{subfiles}
\graphicspath{{\subfix{../figs/}}}
\begin{document}

%*******************************************************
% Abstract
%*******************************************************
%\renewcommand{\abstractname}{Abstract}
\pdfbookmark[1]{Abstract}{Abstract}
% \addcontentsline{toc}{chapter}{\tocEntry{Abstract}}
\begingroup
\let\cleardoublepage\relax
\let\cleardoublepage\relax

\chapter*{Abstract}
Language has a crucial role in and is greatly influenced by widely different spheres of
society, from simple interpersonal communication to the economy and culture. This is
what makes sociolinguistics, the study of the interactions of language and society, a
complex but decidedly worthwhile endeavour. As a wealth of linguistic data can be
retrieved from online social media, the development of new theoretical models aimed at
uncovering mechanisms underlying sociolinguistic phenomena can be better guided and
tested than ever before. In this thesis, we harness this great potential, and take an
interdisciplinary approach to sociolinguistics that is inspired by methods of complex
systems and data science.

First, we study languages as coherent units that compete with others for speakers, in
order to try to identify the drivers of language death and how coexistence of multiple
languages in an interconnected society might come to be. Crucially, we take into account
the spatial embedding of languages, and first observe it using Twitter data. We find
that two languages can coexist with completely separated communities but also with
communities mixed in space, featuring a large population of bilinguals. We capture this
diversity of coexistence states by introducing a model that considers a potential
cultural attachment for one language that may counteract a globally lower prestige, as
well as the relative ease to learn a language knowing the other. Both simulations' and
analytic results are used to support our claims.

We then focus on variation within a language to point out a potential dependence of
standard language use with socio-economic status. Focusing on England, we find that
there is a slight tendency for English Twitter users to make more grammatical mistakes
the lower their income is. This tendency is however very different from one metropolitan
area to another, and actually, it seems to be weaker the more socio-economic classes mix
together. We propose a model that accounts for potentially different mixing patterns and
preferences for a language variety. It reproduces this effect we observed in simple
simulations and more realistic ones.
% TODO: a preciser quand resultats definitifs
We thus find that increased social mixing is crucial to tackle potential social and
economic segregation reflected in this linguistic variation.

Lastly, we leverage the interrelationship between language and culture in a case study
of the United States to define its major cultural regions. From geotagged tweets written
in English, we find the usage hotspots of words found in them to then compute the
principal dimensions of lexical variation. With these, we are able to infer coherent
cultural regions and the topics that define them. This quantitative, automatic analysis
thus provides robust answers to the debate around cultural geography, which has been
historically marked by differing definitions of relevant cultural factors.

The strength of the results we obtained across quite diverse areas of sociolinguistics
is a mirror of the strength of the approach we took throughout our work, that relies on
computational tools, large datasets and simple mathematical models. It calls for further
developments of this kind, which are most probably only in their infancy. 

\clearpage

\begin{otherlanguage}{french}
\pdfbookmark[1]{Résumé}{Résumé}
\chapter*{Résumé}
Le langage a un rôle central et subit de fortes influences venant de sphères très
diverses de la société, et ce de la simple communication entre individus à l'économie et
la culture. C'est cela qui rend la sociolinguistique, c'est-à-dire l'étude des
interactions entre le langage et la société, une entreprise à la fois complexe et
incontestablement digne d'intérêt. Alors qu'une quantité inédite de données peuvent être
extraites des réseaux sociaux, le développement de nouveaux modèles théoriques qui
visent à identifier les mécanismes sous-jacents aux phénomènes sociolinguistiques peut
être mieux guidé et mis à l'épreuve que jamais auparavant. Dans cette thèse, nous tirons
parti de ce potentiel, et prenons une approche interdisciplinaire à la sociolinguistique
inspirée par des méthodes de science des systèmes complexes et de la science des
données.

En premier lieu, nous étudions les langues comme des unités cohérentes qui sont en
compétition les unes avec les autres afin d'essayer d'identifier les principaux facteurs
qui mènent à la mort d'une langue, et ce qui pourrait rendre possible la coexistence de
multiples langues dans une société interconnectée. Un aspect crucial que nous prenons en
compte est l'ancrage géographique des langues, que nous observons à travers des données
de Twitter. Ces observations nous montrent que deux langues peuvent coexister à travers
deux communautés complètement séparées, mais aussi lorsque ces dernières cohabitent,
avec une population considérable de bilingues. Afin de saisir cette diversité d'états de
coexistence, nous introduisons un modèle qui considère la possibilité d'un attachement
culturel à l'une des langues qui pourrait contrebalancer un prestige globalement
inférieur, ainsi que la facilité relative d'apprentissage d'une langue sachant parler
l'autre. Nos résultats découlant à la fois d'analyse mathématique et de simulations
numériques viennent appuyer nos thèses.

Par la suite, nous nous focalisons sur les variations intrinsèques à une langue afin
d'identifier une potentielle dépendance entre le respect des normes standard d'une
langue et le status socio-économique des individus. Nous concentrons notre analyse sur
l'Angleterre et identifions une légère tendance pour les utilisateurs Anglais de Twitter
de commettre plus d'erreurs grammaticales s'ils ont un revenu plus bas. Cette tendance
est en revanche très différente d'une métropole à l'autre, et, de fait, notre analyse
indique qu'elle soit plus faible quand différentes classes sociales se mélangent plus.
Nous proposons alors un modèle qui prend en compte un plus ou moins grand brassage
social et de potentielles préférences de certaines classes pour une variété
linguistique. Il reproduit l'effet que nous avons observé à la fois dans des
configurations très simples qui nous permettent de l'analyser mathématiquement, mais
également dans des simulations plus réalistes à base d'agents. Nos résultats indiquent
donc que plus de brassage social est crucial pour contrecarrer de potentielles
ségrégations économiques et sociales qui se reflètent dans cette variation linguistique.

La dernière étude que nous présentons utilise le caractère indissociable de la relation
entre langage et culture dans une étude de cas des États-Unis afin de définir ses
principales régions culturelles. À partir de tweets géolocalisés écrits en anglais, nous
cartographions les zones où certains mots sont utilisés plus ou moins que de coutume
pour ensuite déterminer les principales dimensions de variation lexicale dans le pays.
Nous pouvons alors déduire de celles-ci les principales régions culturelles et les
sujets qui les définissent. Cette approche quantitative et automatique fournit ainsi des
réponses robustes au débat qui entoure la géographie culturelle, historiquement marquée
par des manières différentes de définir les dimensions culturelles proéminentes.

La force des résultats que nous avons obtenus à travers des domaines assez variés de la
sociolinguistique n'est que le miroir des forces de l'approche que nous avons adoptée
tout au long de cette thèse, qui repose sur des outils computationnels, de grands
ensembles de données et des modèles mathématiques simples. Cela invite donc à de plus
amples études de ce type, qui ne sont probablement qu'à leur genèse.
\end{otherlanguage}


\endgroup

\vfill


\end{document}