\documentclass[../thesis.tex]{subfiles}
\graphicspath{{\subfix{../figs/}}}
\addbibresource{biblio.bib}
\begin{document}

\chapter{Conclusion}
\label{ch:conclusion}

\epigraph{
  \begin{center}
    \textnormal{Guido (Marcello Mastroianni)}\\
  \end{center}
  I'm not afraid anymore of telling the truth, of the things I don't know, what I'm
  looking for and what I haven't found yet. This is the only way I can feel alive and I
  can look into your faithful eyes without shame.
}{
  \epigraphcite{Fellini1963}
}


\section{Summary of our findings}
Throughout this thesis, we have strived to push further the understanding of
sociolinguistic phenomena, by first and foremost always relying on some representative
measurement of them, in particular from social media data. Delving into large amounts of
data allowed us to give quantitative descriptions, but we should and did not stop at
these, and proposed theoretical models inspired by complexity science that are the ones
really that can really push further our understanding, and also lay down a fertile
ground for later developments.

In the first study we presented in \cref{ch:multiling}, we applied this methodological
framework to the study of language diversity and its decline, and the potential for a
recovery. Leveraging geotagged Twitter data from multilingual regions around the globe,
we observed and quantified how the distributions of speakers of different languages can
display widely different spatial arrangements. Indeed, if those were random, people from
different linguistic communities would be spread evenly, and the geographical
distribution of each group should be indistinguishable from the whole. We thus
quantified how these groups were segregated by introducing the \ac{EMR}, which
quantifies the departure from this state. At the two extremes, we found regions in which
language groups are almost completely separated, such as Switzerland or Belgium, and
others in which they mix almost indifferently, such as Catalonia or Galicia. We
subsequently proposed a model that tries to capture the mechanisms behind this
observation. It features two languages, and individuals that can speak one of the two
languages, or both. The transition probabilities from one language group to another do
not only depend on the proportion of speakers of each language, or on the external
prestige attributed by institutions to one of the two languages, that were introduced in
previous studies, but also on the relative ease to learn one language when knowing the
other, and on the preference of bilinguals for one of the two languages. To understand
if and how stable coexistence of languages may emerge within this model, we determined
the stable fixed points of the model in mean-field and set in an interconnected
population. The two parameters we introduced were found to be crucial to pass from
stable solutions of extinction and dominance to ones of coexistence. They even allow to
have a stable community of bilingual sustain a minority language. Then, as there still
remained to understand how different spatial configurations can emerge, we simulated the
model in a metapopulation framework, incorporating the daily commuting of individuals in
Belgium. Remarkably, we found that the present state featuring a boundary between
Flemish and French can be stable if the relative rate of learning the other language
remains low enough. Further, if the latter increases, at some point bilingualism would
spread in the country, as is the case in Catalonia. Still, if it increases, but not
enough, one language may actually spread over the boundary and overshadow the other. Our
modelling effort has thus proved fruitful, as it uncovered that cultural attachment to a
language and policies facilitating language learning can be critical in enabling
language coexistence.

Our second study presented in \cref{ch:ses} is very much similar to the first in terms
of methodology, but it is however applied to a very different issue. This time, we
looked into variation within a language, and in particular into variations away from the
standard form, and how this variation may arise from socio-economic segregation. We
therefore focussed on one language in particular, the English language, in one country,
the UK. We ran a grammar and spell checker on geotagged tweets of more than a hundred
thousand Twitter users for whom we could determine the \ac{MSOA} of residence. This
enabled us an analysis of the correlation between the frequency of grammar mistakes
these users make, and the average income of the area where they live, serving as a
approximate proxy for their \ac{SES}. We found a rather weak but significant trend for
users from poorer areas to make more mistakes. However, zooming in separately on eight
metropolitan areas of England, we spotted very different correlations for the same
variables across these areas. Remarkably, after inferring the broad mobility patterns of
different \ac{SES} classes in these areas from their user's geotagged tweets, we
discovered that these correlations are actually lower the higher their assortativity in
this city. In other words, our results could imply that mixing of different classes
smoothes out linguistic differences between them. We subsequently proposed an \ac{ABM}
that tries to explain this phenomenon. It considers individuals that can belong to one
of two \ac{SES} classes, which is a fixed property, and may speak one of two varieties
of their language, which is subject to change. Similarly to our model of language
competition, it incorporates the prestige of one of the two varieties, as well as the
relative preference of each class for each variant. To account for the effect of
assortativity, individuals have a cell of residence that they share with others of
similar \ac{SES}, and can have different probabilities to move to other cells. In a
simple setting with just two cells and symmetric mobility patterns of the classes, we
have shown through a mathematical analysis of the fixed points of the model set in
mean-field that a strong preference of the lower class to speak non-standard can help
sustain this variety, and that more mixing does indeed push toward more similar
variety adoption across classes.
% TODO: simulations

In \cref{ch:acr}, we again focus on the English language, but this time on the US to
investigate how lexical variation may reflect cultural differences. We proposed an
alternative to the works of cultural geographers who defined US cultural regions based
on qualitative arguments, or on quantitative ones, but in either case based on quite
arbitrary choices in the definition of the important cultural factors. Our alternative
is rather based on inferring both cultural regions and features from what people care to
discuss, here again in particular on Twitter. From billions of geotagged tweets, we
derived the word frequency distributions in all counties of the contiguous US. We
extracted the regional hotspots, if any, in the usage of the \SI{10000}{} most frequent
word forms using a metric of local spatial autocorrelation. In order to focus on
relevant signal and not spurious variations in some words' frequencies, we then
determined the principal dimensions of geographical variation with a \ac{PCA}.
Projecting the autocorrelation matrix along the \acp{PC} thus obtained, we performed a
hierarchical clustering to infer groups of counties that may constitute cultural regions
for different numbers of splits. We found a very strong two-way split, between the
South-East of the country and the rest, mostly defined by a strong influence of
African-American culture. We have shown how the country can be further divided into 5
coherent cultural regions, based on topics as diverse as politics, one's natural
surroundings, spectator sports or local pride. This work has thus shown how culture is
indeed deeply embedded into language, and how this can be leveraged to determine the
factors that are actually important in a people's culture.

% TODO: paragraph on yeah method is nice?

\section{What I learned along the way}

% clarity. from philosophy: first and foremost, define what it is you're talking about. from physics and maths: clearly define the problem before trying to solve it. 

% blabla importance of quantitative aspects and complexity science approach, adapt first sentence of next subsec when this is written

\subsection{Crossing disciplinary boundaries}
Given this approach I took throughout my years of PhD, it is only natural that I ended
up crossing disciplinary boundaries. In practice, this is reflected first in the
different backgrounds of the researchers I had the chance to collaborate with. For
instance, I had the opportunity to collaborate on the work presented in \cref{ch:acr}
with Jack Grieve, whose background is in linguistics. While I was mostly focused on the
quantitative aspects of the work, he brought a level of interpretation of our results
that I did not see at first. Further, this made me aware of a whole new range of
literature and research areas I had never heard of. This is mostly because this
literature was from people outside the field of complexity science, and consequently
published in different journals.

Second, and as just said above, this is partly due to the first point: this
cross-disciplinarity is apparent from the diversity of literature cited throughout this
thesis. I can illustrate how twisted navigating this diversity can be, but how rewarding
it is with the example of how I stumbled upon a metric such as the \ac{EMD}, that I use
in \cref{ch:multiling}.
% came to me from: I need some kind of distance between two probability distributions embedded in a metric space.
% chequerboard problem \cite{WhiteMeasurementSpatial1983}
First finding out the shortcomings of spatial entropies, widely accepted in the field of
spatial complex systems \cite{BattyEntropyComplexity2014} It has mainly been used within
the field of computer vision \cite{RubnerMetricDistributions1998}, and it was shown to
be a proper distance (in the metric sense) between probability distributions
\cite{LevinaEarthMover2001}. Nothing but a particular version of the Wasserstein or
Mallows distance, well known in transport theory. Only to find out a year after writing
this article that the geographer
\citefirstlastauthor{JakubsDistancebasedSegregation1981} had defined a metric equivalent
to the \ac{EMR} in 1981 \cite{JakubsDistancebasedSegregation1981}, much before computer
vision had the occasion to exploit it. Natural that different fields end up using
different terms. In this case, mathematicians have defined a general distance, which is
used in other applied fields within a particular case. So the mathematicians need this
more general name. And would anyone really think that a geographer or engineer in
computer vision should give up the very evocative \ac{EMD} in favour of ``the
Wasserstein 1-distance on discrete empirical distributions''? So what can be done to
bridge these gaps? Mostly curiosity that drove me to find all of this out. One should
not be afraid to search for literature outside their field, and to simply acknowledge
them. Citing a work outside your field is much more valuable than citing another
everyone within your field already knows, as they are probably going to be most of your
readers.
% reviews not of a field but of an idea nad how it's being used across disciplines

All in all, the people and the research you interact with are highly interdependent. And
as I have shown, more diversity in those can be the foundation for more throrough and
further-reaching research. While desirable, increasing this diversity in one's research
is far from straightforward. Beyond the point made above that looking for and citing
other fields' literature can help, or the obvious that one should try to go to different
conferences or make project proposals with an actually interdisciplinary team, I would
like to highlight the lack of interdisciplinary reviews. Review articles are the first
literature one seeks to enter into a field. Yet, the notion of field itself limits the
scope of reviews such as \cite{CastellanoStatisticalPhysics2009}, which is a very good
review of physics-inspired models of social phenomena, but exactly as such, it targets
almost exclusively researchers with a background in physics. While there is no doubt
this kind of review adds great value, another kind that would focus rather on an issue
that is tackled by multiple fields, and that would review their different approaches and
confront or relate them, would also be beneficial. For instance, instead of answering
the question ``what physics-inspired models have been used to study social dynamics?'',
they would answer one such as ``what is the state of research on language competition
and multilingualism, from the (mostly) qualitative models of linguists to the
quantitative ones of physicists, also considering empirical works?''. This would provide
scientists from different fields insights into  what others have to say about a
question, and who they might collaborate with to further everyone's understanding. The
more recent \cite{BoissonneaultSystematicInterdisciplinary2021} is the closest to being
this kind of interdisciplinary review that I am aware of, although more is necessary as
it does not confront the conclusions of the research coming from different fields.


\subsection{Science and trends}
% At the end of these roughly 3 years and a half, there is one thing I am particularly
% proud of: the word ``COVID'' does not appear once in any of my works. Some may say they
% heard society's cry for help and flew to its rescue. While some may have had this kind
% of genuine intentions, I remain quite sceptical that everyone who jumped on the
% scientific hype train had them. Some may seem more like churning out more material for
% the scientific publishing machine. But the truth is, I do not know anything about
% epidemiology, and it does not interest me at all.

The number of researchers interested in language competition models like the one we
presented in \cref{ch:multiling} is very low. The hype has passed, it was 20 years ago
when \citeauthor{AbramsModellingDynamics2003} published a piece in \textit{Nature}. Yet
the problem has not been solved at all, as we have shown ourselves. 



\section{Uncharted directions worth exploring}
While I believe the results summarised above provide valuable insights to
sociolinguistic matters, they leave many questions unanswered, and maybe even raise more
questions than answers. In the following I will list some potential directions for
future research that could build on top of the work we have presented here. First are
unresolved questions in sociolinguistics that naturally arise from the results we have
presented.

For instance, can the model of language shift we presented in \cref{ch:multiling} be
equally applied to diachronic data? A dataset with historical data spanning centuries
was for example used recently in \cite{SeoaneAreDutch2022}. One could try to see if our
model can be fit to these data, and, if not, what ingredients may be missing. A
comparison with the results obtained in \cite{SeoaneAreDutch2022} with a different model
can also bring valuable insights to the field. 

Another idea about modelling language competition that is worth looking into and that we
have not mentioned yet is to consider language as a property not only of the individuals
but also of their interactions, as suggested in \cite{CarroCoupledDynamics2016}. Indeed,
as shown in this article, a language might be preserved within a tight-knit community
that uses it internally, while being able to switch to another to communicate with
others. This applies to language shift models of \cref{ch:multiling}, but also to models
of usage of a language's varieties, as the one of \cref{ch:ses_ling}. What is missing
here is more empirical work on the question. This would imply a considerable effort, as
obtaining a network of interactions along with the language in which they take place
from real-world data is very challenging. But this direction is theoretically very
promising, so these efforts to put together such a dataset may well be worthwhile.

The central idea behind the inference of American cultural regions we presented in
\cref{ch:acr} can also lead to many further developments. One could look at countries
that share a language and uncover their cultural differences. Interesting questions can
then be investigated. Can some areas of a country be more similar to another country's
than their own? It would also be interesting to see how the colonial past of some
countries is reflected in those. More interesting questions then arise. Did more
conflictual decolonisation processes lead to starker cultural differences? Or, does a
more recent split imply more similarity?

\Ac{SES} may very well impact other aspects of people's speech than whether one abides
by the rules defining the standard form of their language in writing. The persistence of
different accents or some lexical items may also be linked to it, and in particular to
the preferential attachment of the low \ac{SES} class to them. 

The second kind of directions I would like to suggest are simply interesting questions
that arose while I worked on this thesis, as, while investigating matters of
sociolinguistics, we stumbled upon interesting issues in social science that are not
confined to the realm of linguistics.

One is segregation. This theme permeates \cref{ch:multiling},
\cref{ch:ses_ling}, and also \cref{ch:acr} to a lesser extent. But these are
segregations of different kinds: linguistic, cultural and socio-economic. A question
that remains very much open is how these types of segregation are interrelated, also
taking into account the segregation in social contacts, also known as assortativity or
homophily.

Another is the impact of cultural differences on social behaviours in general. What
features matter the most when building social groups? How does this change from one area
of the world to another?

Both models presented in \cref{ch:multiling} and \cref{ch:ses_ling} integrate effects of
social pressure. They simply assume that the most people around you speaks a language or
a variety, the more likely you are to adopt it too. Psychologists have argued about the
existence of a so-called ``psychosocial law'', which states that the $N^\text{th}$
individual you interact with that displays some behaviour has less impact on you than
the $(N - 1)^\text{th}$, which is equivalent to having a sublinear functional dependence
of social impact on the proportion of different agents ($a < 1$ in the models presented
in \cref{ch:multiling}). Some empirical works using controlled experiments seem to
confirm this theory, however there are also researchers arguing for the existence of a
threshold effect in social contagion, made popular through the concept of social tipping
point \cite{MilkoreitDefiningTipping2018}. It would be very interesting to bridge the
understanding between these works on social pressure and, for instance, all the works
studying social contagion in complexity science \cite{WattsInfluentialsNetworks2007},
including the more recent investigations into higher-order effect
\cite{IacopiniSimplicialModels2019}.

There are thus many more directions worth exploring, and, for now, to most questions the
most sensible answer I can provide is: I don't know. But while one could consider
knowledge as comforting some faith in the universe, I believe uncertainty to be much
more entertaining. Let us then keep asking questions to keep things that way.
% But while at first the realisation that one cannot know everything might make one's lose faith in the universe.
At last, it may seem that the quest for understanding paradoxically leads to less
understanding, but not to worry:
as the poet would say,
% to quote some necessarily poorly translated French poetry,
% \emph{c'qui compte c'est pas l'arrivée, c'est la quête}
% as the poet would say,
it's the quest itself that matters, not the destination.
% it's not the destination that matters, it's the quest.

\end{document}
