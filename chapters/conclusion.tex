\documentclass[../thesis.tex]{subfiles}
\graphicspath{{\subfix{../figs/}}}
\addbibresource{biblio.bib}
\begin{document}

\chapter{Conclusion}
\label{ch:conclusion}

\epigraph{
  \begin{center}
    \textnormal{Guido (Marcello Mastroianni)}\\
  \end{center}
  I'm not afraid anymore of telling the truth, of the things I don't know, what I'm
  looking for and what I haven't found yet. This is the only way I can feel alive and I
  can look into your faithful eyes without shame.
}{
  \epigraphcite{Fellini1963}
}


\section{Summary of our findings}
Throughout this thesis, we have strived to push further the understanding of
sociolinguistic phenomena, by first and foremost always relying on some representative
measurement of them, in particular from social media data. Delving into large amounts of
data allowed us to give quantitative descriptions, but we should and did not stop at
these, as we proposed theoretical models inspired by complexity science that are the
ones that can really push further our understanding, and also lay down a fertile ground
for later developments.

In the first study we presented in \cref{ch:multiling}, we applied this methodological
framework to the study of language diversity and the potential for a recovery from its
current decline. Leveraging geotagged Twitter data from multilingual regions around the
globe, we observed and quantified how the distributions of speakers of different
languages can display widely different spatial arrangements. Indeed, if those were
random, people from different linguistic communities would be spread evenly, and the
geographical distribution of each group should be indistinguishable from the whole. We
thus quantified how these groups were segregated by introducing the \ac{EMR}, which
quantifies the departure from this theoretical unsegregated scenario. At the two
extremes, we found regions in which language groups are almost completely separated,
such as Switzerland or Belgium, and others in which they mix almost indifferently, such
as Catalonia or Galicia. We subsequently proposed a model that tries to capture the
mechanisms behind this observation. It features two languages, and individuals that can
speak one of the two languages, or both. The transition probabilities from one language
group to another do not only depend on the proportion of speakers of each language, or
on the external prestige attributed by institutions to one of the two languages, that
were introduced in previous studies, but also on the relative ease to learn one language
when knowing the other, and on the preference of bilinguals for one of the two
languages. To understand if and how stable coexistence of languages may emerge within
this model, we determined the stable fixed points of the model in mean-field and set in
a completely interconnected population. The two parameters we introduced were found to
be crucial to pass from stable solutions of extinction and dominance to ones of
coexistence. They even allow to have a stable community of bilinguals sustain a minority
language. Then, as there still remained to understand how different spatial
configurations can emerge, we simulated the model in a metapopulation framework,
incorporating the daily commuting of individuals in Belgium. Remarkably, we found that
the present state featuring a boundary between Flemish and French can be stable if the
relative rate of learning the other language remains low enough. Further, if this rate
were to increase, at some point bilingualism would spread across the country, as is the
case in Catalonia. Still, if it were to increase, but not enough, one language may
actually spread over the boundary and overshadow the other. Our modelling effort has
thus proved fruitful, as it uncovered that cultural attachment to a language and
policies facilitating language learning can be critical in enabling language
coexistence.

Our second study presented in \cref{ch:ses} is very much similar to the first in terms
of methodology, but it is however applied to a very different issue. This time, we
looked into variation within a language, and in particular into variations away from the
standard form, and how this variation may arise from socio-economic segregation. We
therefore focussed on one language in particular, the English language, in one country,
the UK. We ran a grammar and spell checker on geotagged tweets of more than a hundred
thousand Twitter users for whom we could determine the \ac{MSOA} of residence. This
enabled us an analysis of the correlation between the frequency of grammar mistakes
these users make, and the average income of the area where they live, which serves as a
proxy for their \ac{SES}. We found a rather weak but significant trend for users from
poorer areas to make more mistakes. However, zooming in separately on eight metropolitan
areas of England, we spotted very different correlations for the same variables across
these areas. Remarkably, after inferring the broad mobility patterns of different
\ac{SES} classes in these areas from their user's geotagged tweets, we discovered that
these correlations are actually lower the higher their assortativity in this city. In
other words, our results could imply that mixing of different classes smoothes out
linguistic differences between them. We subsequently proposed an \ac{ABM} that tries to
explain this phenomenon. It considers individuals that can belong to one of two \ac{SES}
classes, which is a fixed property, and may speak one of two varieties of their
language, which is subject to change. Similarly to our model of language competition, it
incorporates the prestige of one of the two varieties, as well as the relative
preference of each class for each variant. To account for the effect of assortativity,
individuals have a cell of residence that they share with others of similar \ac{SES},
and can have different probabilities to move to other cells. In a simple setting with
just two cells and symmetric mobility patterns of the classes, we have shown through a
mathematical analysis of the fixed points of the model set in mean-field that a strong
preference of the lower class to speak non-standard can help sustain this variety, and
that more mixing does indeed push toward more similar variety adoption across classes.
% TODO: simulations

In \cref{ch:acr}, we again focus on the English language, but this time on the US to
investigate how lexical variation may reflect cultural differences. We proposed an
alternative to the works of cultural geographers who defined US cultural regions based
on qualitative arguments, or on quantitative ones, but in either case based on quite
arbitrary choices in the definition of the important cultural factors. Our alternative
is rather based on inferring both cultural regions and their features from what people
care to discuss, here again in particular on Twitter. From billions of geotagged tweets,
we derived the word frequency distributions in all counties of the contiguous US. We
extracted the regional hotspots, if any, in the usage of the \SI{10000}{} most frequent
word forms using a metric of local spatial autocorrelation. In order to focus on
relevant signal and not spurious variations in some words' frequencies, we then
determined the principal dimensions of geographical variation with a \ac{PCA}.
Projecting the autocorrelation matrix along the \acp{PC} thus obtained, we performed a
hierarchical clustering to infer groups of counties that may constitute cultural regions
for different numbers of splits. We found a very strong two-way split, between the
South-East of the country and the rest, mostly defined by a strong influence of
African-American culture. We have shown how the country can be further divided into 5
coherent cultural regions, based on topics as diverse as politics, one's natural
surroundings, spectator sports, or local pride. This work has thus shown how culture is
indeed deeply embedded into language, and how this can be leveraged to determine the
factors that are actually important in a people's culture.

% TODO: paragraph on yeah method is nice?

\section{What I learned along the way}

\subsection{The state of open data}
A perceptive reader will have noticed from the last section that Twitter data has been
central in every work I have presented here. Therefore, if not for the continued effort
made by technicians of my institute to collect Twitter data over more than seven years,
most of the research presented here would never have come to light. This work has thus
been made possible first thanks to these technicians' work, to whom I am grateful, but
also fundamentally because Twitter allowed it. And no one, me included, can deny that
having so much research rely on a single, private source of data, is highly problematic.
Indeed, counting on the good will of private online platforms is not a viable foundation
for research \cite{AusloosOperationalizingResearch2020}. As we already mentioned in
\cref{sec:methods_online_data}, since private companies are driven by economic growth
and (sometimes) profit, they have no incentive to contribute to academic research, apart
maybe for some marginal publicity. Hence why platforms that are generous in the data
they provide to researchers through their \ac{API} are few. And they seem to be getting
fewer, as recent announcements about academic access to Twitter \ac{API} seem to
indicate. One could object that researchers could still pay for access to these data to
make it an agreeable deal. But this still poses the problem of the power these private
entities could hold over public, a priori independent research. For instance, for as
long as it has existed, academic access to the Twitter \ac{API} has been free but
subject to their approval, after an internal review based on opaque criteria. If they
wished to do so, they had every right to revoke the access of a researcher. This
threatens the independence of the researchers that may wish to study these platforms,
through a potential censorship by the platforms, but also any induced self-censorship of
the researcher, whether conscious or not. It is hard to imagine Twitter censoring the
kind of works I have presented here, but for instance, people who are interested in the
inner workings of the platform and its potential harmfulness to society could be
concerned.

There is potential for public regulation to tackle this issue. A first step in the right
direction may very well be in the making as the Digital Services Act of the European
Union is coming into force, in particular its article 40 \cite{DigitalServices2022}. It
aims at facilitating independent auditing of very popular online media, for instance to
quantify the effects their algorithms or potential moderation may have on public debate
or political polarisation. To reach this goal, it outlines a process that allows vetted
European researchers to send data requests to a European institution for them to
evaluate their pertinence, and if they approve it, to forward the request to the
designated platform, that is legally bound to comply reasonably well with the request.
The research topics that fall within the scope of this article is thus quite limited,
but it constitutes a great step forward as a framework for academic data access mediated
by public authorities. If successful, one can imagine it extended to other kinds of
companies such as cell phone companies or online retailers
\cite{PersilyProposalResearcher2021}. If a robust framework is laid out to vet research
projects that can give guarantees that the researchers will comply with privacy laws,
and keep the data they were handed within the academic realm, private companies may be
less reluctant if future legislation extends the scope of research that can make such
data requests. This thesis has demonstrated how online social media data can be
extremely valuable to sociolinguistic study, but many other subjects of public
importance can benefit. One can take the example of the study of mobility patterns, for
which data-rich companies such as Google or mobile network operators exceptionally
shared data for studies related to the latest COVID pandemic
\cite{AguilarImpactUrban2022,GozziEstimatingEffect2021}. They were shared with a few
research groups only, but these kinds of precedents show trust can be built between
private companies and researchers. Having public authorities mediate between the two has
the potential to democratise the access of the research community to the very valuable
datasets these companies hold. There only remains to see how well these new rules will
actually be implemented, and on this also depends whether other countries would be
inclined to follow suit.
% A major problem then seems to be trust

% but also to enable independent
% auditing, that platforms have repeatedly failed to comply with
% \cite{AusloosOperationalizingResearch2020}.

% mention do not know if individuals actively sharing thier data (ref contact tracing
% apps) is realistic. stuff like mobility may be because everyone has a phone with GPS,
% could just store their location history locally and share it for research (b/c anyway
% it's shared with Google ha). but speech production like this thesis? communication
% necessarily goes through a service that's going to be operated by a company, or maybe
% decentralised instances (ref fediverse). theoretically possible but lots of questions
% to answer, complex problem. SO would be cool for this thing to be opt in, but won't
% happen tomorrow

% An alternative can also be to obtain data from people themselves, and not from the
% companies that collect them from them. While this is much more desirable as it is much
% more democratic and  ndividual data ownership .

% Again, trust is of the essence. Only if robust, transparent processes are put into place
% will the general public be willing to share their data for research purposes.
% Decentralised platforms in which users have much more power over their own data. Example
% of Fediverse, can review servers' policies, and if they end up changing it, can just
% easily switch servers without losing anything. If they wish to make their data available
% for researchers, should be easy way to do so. Decentralised nature makes the data
% request and access more complex though.


\subsection{Crossing disciplinary boundaries}
Given the approach I have taken throughout my years of PhD, it is only natural that I
ended up crossing disciplinary boundaries. In practice, this is reflected first in the
different backgrounds of the researchers I had the chance to collaborate with. For
instance, I had the opportunity to collaborate on the work presented in \cref{ch:acr}
with Jack Grieve, whose background is in linguistics. While I was mostly focused on the
quantitative aspects of the work, he brought a level of interpretation of our results
that I did not see at first. Further, this made me aware of a whole new range of
literature and research areas I had never heard of. This is mostly because this
literature was from people outside the field of complexity science, and consequently
published in different journals.

Second, and this is partly due to the point made above: this cross-disciplinarity is
apparent from the diversity of literature cited throughout this thesis. I can illustrate
how twisted navigating this diversity can be, but how rewarding it is with the example
of how I stumbled upon a metric such as the \ac{EMD}, that I use in \cref{ch:multiling}.
For a brief context, in this work we wanted to give a quantitative estimate of how
segregated different language groups were in some multilingual regions. 
% Since traditional metrics for the field of complexity science, such as entropy
One way to conceptualise a metric giving this kind of estimate is as a distance between
two spatially embedded distributions: the one of the total population, and the one of
the language group under consideration. 
% % chequerboard problem \cite{WhiteMeasurementSpatial1983}
% First finding out the shortcomings of spatial entropies, widely accepted in the field of
% spatial complex systems \cite{BattyEntropyComplexity2014}
It was with this kind of formulation in mind that I stumbled upon the \ac{EMD} as used
within the field of computer vision \cite{RubnerMetricDistributions1998}. Indeed, the
\ac{EMD} is nothing but a particular version of the Wasserstein or Mallows distance,
well known in transport theory, and is thus a proper distance (in the metric sense)
between probability distributions, as shown in \cite{LevinaEarthMover2001}. As it turns
out, I found out roughly a year after writing the article linked to this chapter that
the geographer \citefirstlastauthor{JakubsDistancebasedSegregation1981} had defined a
metric equivalent to the \ac{EMR} in 1981 \cite{JakubsDistancebasedSegregation1981},
much before the term \ac{EMD} was discussed in the realm of computer vision. With a
different formulation of the problem, he came up with a metric very similar to mine.
Only it has most of the time only been mentioned in passing in posterior works and never
applied in large scale studies --- due to computation limitations more than theoretical
flaws, I suspect. This dive into literature from a mix of computer vision, mathematics
and geography, in the completely wrong order, has shown me how researchers from
different fields can talk about the same concepts with completely different words, and
also arrive to the same conclusions with different trains of thoughts.
% Natural that different fields end up using different terms. In this case,
% mathematicians have defined a general distance, which is used in other applied fields
% within a particular case. So the mathematicians need this more general name. And would
% anyone really think that a geographer or engineer in computer vision should give up
% the very evocative \ac{EMD} in favour of ``the Wasserstein 1-distance on discrete
% empirical distributions''? So what can be done to bridge these gaps?
That is why, among other reasons, I thus believe being curious about what ``other
fields'' do and how they conceptualise problems is definitely worth the invested time.
% One should not be afraid to search for literature outside their field, and to
% simply acknowledge them.
Further, citing a work outside your field is much more valuable than citing another
everyone within your field already knows, as most of your readers are probably going to
be from your field.

All in all, the people and the research one interacts with are highly interdependent.
And, as I think this thesis shows, more diversity in those can be the foundation for
more thorough and further-reaching research. But while desirable, increasing this
diversity in one's research is far from straightforward. Beyond the point made above
that looking for and citing other fields' literature can help, or the obvious that one
should try to go to different conferences, or pursue projects with actually
interdisciplinary teams, I would like to highlight the lack of interdisciplinary
reviews. Review articles are the first literature one seeks to enter into a field. Yet,
the notion of field itself limits the scope of reviews such as
\cite{CastellanoStatisticalPhysics2009}, which, while a very good review of
physics-inspired models of social phenomena, exactly as such it targets almost
exclusively researchers with a background in physics. While there is no doubt this kind
of review adds great value, I believe another kind that would focus rather on an issue that is
tackled by multiple fields, and that would review their different approaches and
confront or relate them, would also be beneficial. For instance, instead of answering
the question ``what physics-inspired models have been used to study social dynamics?'',
they would answer one such as ``what is the state of research on language competition
and multilingualism, from the (mostly) qualitative models of linguists to the
quantitative ones of physicists, also considering empirical works?''. This would provide
scientists from different fields insights into what others have to say about a
question, and who they might collaborate with to further everyone's understanding. The
more recent \cite{BoissonneaultSystematicInterdisciplinary2021} is the closest to being
this kind of interdisciplinary review that I am aware of, although more is necessary as
it does not confront the conclusions of the research coming from different fields.


\section{Uncharted directions worth exploring}
While I believe the results summarised above provide valuable insights to
sociolinguistic matters, they leave many questions unanswered, and may actually even
raise more new questions than answers. In the following, I will list some potential
directions for future research that could build on top of the work we have presented
here. First are unresolved questions in sociolinguistics that naturally arise from the
results we have presented.

For instance, we studied only the stable fixed points of the model of language shift
presented in \cref{ch:multiling}. But could it be equally applied to diachronic data? A
dataset with historical data from Belgium spanning centuries was for example used
recently in \cite{SeoaneAreDutch2022}. One could try to see if our model can be fit to
these data, and, if not, what ingredients may be missing. A comparison with the results
obtained in \cite{SeoaneAreDutch2022} with a different model can also bring valuable
insights to the field. 

Another idea about modelling language competition that is worth looking into and that we
have not mentioned yet is to consider language as a property not only of the individuals
but also of their interactions, as suggested in \cite{CarroCoupledDynamics2016}. Indeed,
as shown in this article, a language might be preserved within a tight-knit community
that uses it internally, while being able to switch to another to communicate with
others. This applies to language shift models of \cref{ch:multiling}, but also to models
of usage of a language's varieties, as the one of \cref{ch:ses_ling}. What is missing
here is more empirical work on the question. This would imply a considerable effort, as
obtaining a network of interactions along with the language in which they take place
from real-world data is very challenging. But this direction is theoretically very
promising, so these efforts to put together such a dataset may well be worthwhile.

The central idea behind the inference of American cultural regions we presented in
\cref{ch:acr} can also lead to many further developments. One could look at countries
that share a language and uncover their cultural differences. Interesting questions can
then be investigated. Can some areas of a country be more similar to another country's
than their own? It would also be interesting to see how the colonial past of some
countries is reflected in those. More interesting questions then arise. Did more
conflictual decolonisation processes lead to starker cultural differences? Or, does a
more recent split imply more similarity?

On another topic, \ac{SES} may very well impact other aspects of people's speech than
whether one abides by the rules defining the standard form of their language in writing.
The persistence of different accents or some lexical items may also be linked to it, and
in particular to the preferential attachment of the low \ac{SES} class to them. Are the
dynamics for those linguistic features similar to those we have described in
\cref{ch:ses_ling}? That is another question we cannot answer for the moment.

The second kind of directions I would like to suggest are simply interesting questions
that arose while I worked on this thesis, as, while investigating matters of
sociolinguistics, we stumbled upon interesting issues in social science that are not
confined to the realm of linguistics.

One is segregation. This theme permeates \cref{ch:multiling},
\cref{ch:ses_ling}, and also \cref{ch:acr} to a lesser extent. But these are
segregations of different kinds: linguistic, cultural and socio-economic. A question
that remains very much open is how these types of segregation are interrelated, also
taking into account the segregation in social contacts, also known as assortativity or
homophily.

Another is the impact of cultural differences on social behaviours in general. What
features matter the most when building social groups? How does this change from one area
of the world to another?

Both models presented in \cref{ch:multiling} and \cref{ch:ses_ling} integrate effects of
social pressure. They simply assume that the most people around you speaks a language or
a variety, the more likely you are to adopt it too. Psychologists have argued about the
existence of a so-called ``psychosocial law'', which states that the $N^\text{th}$
individual you interact with that displays some behaviour has less impact on you than
the $(N - 1)^\text{th}$, which is equivalent to having a sublinear functional dependence
of social impact on the proportion of different agents ($a < 1$ in the models presented
in \cref{ch:multiling}). Some empirical works using controlled experiments seem to
confirm this theory, however there are also researchers arguing for the existence of a
threshold effect in social contagion, made popular with the concept of social tipping
points \cite{MilkoreitDefiningTipping2018}. It would be very interesting to bridge the
understanding between these works on social pressure and, for instance, all the works
studying social contagion in complexity science \cite{WattsInfluentialsNetworks2007},
including the more recent investigations into higher-order effect
\cite{IacopiniSimplicialModels2019}.

There are thus many more directions worth exploring, and, for now, to most questions
raised in the above, and to many others I have not yet been able to formulate, the most
sensible answer I can provide is: I don't know. But while one could consider knowledge
as comforting some faith in the universe, I believe uncertainty to be much more
entertaining. Let us then keep asking questions to keep things that way.
% But while at first the realisation that one cannot know everything might make one's lose faith in the universe.
At last, it may seem that the quest for understanding paradoxically leads to less
understanding, but not to worry:
as the poet would say,
% to quote some necessarily poorly translated French poetry,
% \emph{c'qui compte c'est pas l'arrivée, c'est la quête}
% as the poet would say,
it's the quest itself that matters, not the destination.
% it's not the destination that matters, it's the quest.

\end{document}
