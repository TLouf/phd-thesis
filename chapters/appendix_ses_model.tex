\documentclass[../thesis.tex]{subfiles}
\graphicspath{{\subfix{../figs/}}}
\begin{document}

\chapter{Analytic results for the model of variety adoption}
\label{ch:appendix_ses_model}

\section*{Notation}
Let us introduce the following notation:
\begin{itemize}
    \item $S$ the set of \ac{SES} classes in a population
        \begin{equation}
            S = \left\{ s_k \mid k \in [1, n_S] \right\},
        \end{equation}
    \item $C$ the set of cells of residence:
        \begin{equation}
             C = \left\{ c_i \mid i \in [1, n_C] \right\},
        \end{equation}
    \item $N_{c, s}$ the number of residents of cell $c$ with class $s$,
    \item $N_{c} \equiv N_{c, \cdot}$ the population of cell $c$,
    \item $N_{s} \equiv N_{\cdot, s}$ the population of class $s$,
    \item $N \equiv N_{\cdot, \cdot}$ the total population,
    \item $M_{i, j}$ the probability for a resident of $c_i$ to move to $c_j$.
\end{itemize}


\section*{Assumptions}
Let there be only two cells: $n_C = 2$, and two \ac{SES} classes: $n_S = 2$, completely
separated, with the whole $s_1$ population in $c_1$ and the whole $s_2$ population in
$c_2$:
\begin{equation}
    \begin{aligned}
        N_1 \equiv N_{c_1, s_1} = N_{s_1}, \\
        N_2 \equiv N_{c_2, s_2} = N_{s_2}.
    \end{aligned}
\end{equation}
This implies that the $M_{i,j}$ can be summarized with just two values, each
corresponding to a class:
\begin{equation}
    \begin{aligned}
        M_1 \equiv M_{1, 2} = 1 - M_{1, 1}, \\
        M_2 \equiv M_{2, 1} = 1 - M_{2, 2}.
    \end{aligned}
\end{equation}
Let us consider two varieties $v_1$ and $v_2$. This could be the use of standard
language ($v_1$ means they do, $v_2$ means they don't). Now let us introduce an
intrinsic prestige $l_v$ for the variant 2, such that:
\begin{equation}
    \begin{aligned}
        P(1 \rightarrow 2) &\propto l_v \\
        P(2 \rightarrow 1) &\propto 1 - l_v.
    \end{aligned}
\end{equation}
Without loss of generality, let us assume $l_v > 1/2$, meaning 2 is more prestigious
than 1. And let us introduce an asymmetric attachment of each group for their own
variant, $q_1$ and $q_2$:
\begin{equation}
    \begin{aligned}
        P(2 \rightarrow 1 \mid s = s_1) &\propto q_1 > 1/2, \quad&  P(1 \rightarrow 2 \mid s = s_1) \propto 1 - q_1, \\
        P(1 \rightarrow 2 \mid s = s_2) &\propto q_2, \quad&  P(2 \rightarrow 1 \mid s = s_2) \propto 1 - q_2.
    \end{aligned}
\end{equation}
So 2 is more prestigious than 1 but $s_1$ prefers 1.
% TODO: keep > 1/2 stuff?

\section*{Deriving the master equations}
We can then write the following transition probabilities:
\begin{equation}
    \label{eq:first_trans_probs}
    \begin{aligned}
        % v and c depend on t
        P(1 \rightarrow 2 \mid s = s_1, c_t = c_j)
            &= l_v (1 - q_1) P(v_{t-1} = 2 \mid c_t = c_j)
        \\
        P(1 \rightarrow 2 \mid s = s_2, c_t = c_j)
            &= l_v q_2 P(v_{t-1} = 2 \mid c_t = c_j)
        \\
        P(2 \rightarrow 1 \mid s = s_1, c_t = c_j)
            &= (1 - l_v) q_1 P(v_{t-1} = 1 \mid c_t = c_j)
        \\
        P(2 \rightarrow 1 \mid s = s_2, c_t = c_j)
            &= (1 - l_v) (1 - q_2) P(v_{t-1} = 1 \mid c_t = c_j)
    \end{aligned}
\end{equation}
with $P(1 \rightarrow 2) \equiv P(v_t = 2 \mid v_{t-1} = 1)$, $P(2 \rightarrow 1) \equiv
P(v_t = 1 \mid v_{t-1} = 2)$, $t$ denoting the current time step, $P(v_{t-1} = v \mid
c_t = c_j)$ the probability to pick an individual who had variant $v$ at $t-1$ who is at
cell $c_j$ at $t$. Decomposing it by \ac{SES} class, and using Bayes' rule, we get:
\begin{equation}
    \begin{aligned}
        &P(v_{t-1} = v \mid c_t = c_j)
        \\[1ex]
        &
        \begin{aligned}
            & = \sum_{k \in S} P(v_{t-1} = v, s = s_k \mid c_t = c_j) \\
            &= \sum_{k \in S} P(v_{t-1} = v, s = s_k)
                \cdot \frac{P(c_t = c_j \mid v_{t-1} = v, s = s_k)}{P(c_t = c_j)} \\
            &= \sum_{k \in S} P(v_{t-1} = v \mid s = s_k)
                \cdot P(s = s_k)
                \cdot \frac{P(c_t = c_j \mid v_{t-1} = v, s = s_k)}{P(c_t = c_j)}. \\
        \end{aligned}
    \end{aligned}
\end{equation}
Let us introduce $p_{v, s_k} \equiv P(v_{t-1} = v \mid s = s_k)$ to simplify equations
further, and that summarizes the state of the system at the previous step. Also, since
they satisfy $\sum_v p_{v, s_k} = 1$, we will only write in terms of $p_1 \equiv p_{1,
1}$ and $p_2 \equiv p_{2, 2}$. We also have $P(s = s_k) = \frac{N_{s_k}}{\sum_k
N_{s_k}}$. The final term in the product is related to the mobility of the
different SES classes. Indeed, using the fact that $c_t = c_j \mid s = s_k$ is
independent from $v_{t-1} = v$, and aligning the indices of the SES class with the one
of their cell of residence, we have
\begin{equation}
\label{eq:mob_by_SES}
    P(c_t = c_j \mid v_{t-1} = v, s = s_k) = P(c = c_j \mid s = s_k) = M_{k, j},
\end{equation}
and
\begin{equation}
    P(c_t = c_j) = \sum_k P(c = c_j \mid s = s_k) P(s = s_k) = \sum_k M_{k, j} \frac{N_{s_k}}{\sum_k
    N_{s_k}}.
\end{equation}
% and $\sum_j m_{k, j} = 1$.
Let us further introduce
\begin{equation}
    m_{k, j} \equiv \frac{N_{s_k} M_{k, j}}{\sum_{k'} N_{s_{k'}} M_{k', j}},
\end{equation}
which satisfy $\sum_k m_{k, j} = 1$ (unlike the $M_{k, j}$ that satisfy $\sum_j M_{k, j}
= 1$). It is simply the expected proportion of individuals that will appear in $j$ that
have status $k$. We can then write
\begin{equation}
\label{eq:decomp_prev_by_SES}
    P(v_{t-1} = v \mid c_t = c_j) = \sum_k p_{v, s_k} m_{k, j}
\end{equation}

We will further abuse notation and only use $m_1 \equiv
m_{1,2}$ and $m_2 \equiv m_{2, 1}$, which summarize how mobile each group is. Let us then rewrite
\eqref{eq:decomp_prev_by_SES} in terms of $m_1$, $m_2$, $p_1$ and $p_2$ only:
\begin{equation}
    \label{eq:prev_step_mob}
    \begin{aligned}
        P(v_{t-1} = v_2 \mid c = c_1)
            = (1 - m_2) (1 - p_1) + m_2 p_2
        \\
        P(v_{t-1} = v_2 \mid c = c_2)
            = m_1 (1 - p_1) + (1 - m_1) p_2
        \\
        P(v_{t-1} = v_1 \mid c = c_1)
            = (1 - m_2) p_1 + m_2 (1 - p_2)
        \\
        P(v_{t-1} = v_1 \mid c = c_2)
            = m_1 p_1 + (1 - m_1) (1 - p_2)
    \end{aligned}
\end{equation}
We can them write the forms in \eqref{eq:first_trans_probs} in terms of these 4
variables only. But what we actually want is to write $P(v_t = v \mid v_{t-1} = v', s
= s_k)$ for $v' \neq v$. Decomposing this one by cell, we can get the following:
\begin{equation}
    \label{eq:trans_probs_cell_decomp}
    \begin{aligned}
        & P(v_t = v \mid v_{t-1} = v', s = s_k)
        \\[1ex]
        &
        \begin{aligned}
            \quad & = \sum_j P(c = c_j \mid s = s_k)
            \cdot P(v_t = v \mid v_{t-1} = v', s = s_k, c_t = c_j)
        \\
            & = \sum_j M_{k, j} P(v_t = v \mid v_{t-1} = v', s = s_k, c_t = c_j)
        \end{aligned}
    \end{aligned}
\end{equation}
Finally, inserting into \eqref{eq:trans_probs_cell_decomp} the result of the insertion
of \eqref{eq:prev_step_mob} into \eqref{eq:first_trans_probs}, we get:
\begin{equation}
    \label{eq:final_trans_probs}
    \begin{aligned}
        P(1 \rightarrow 2 \mid s = s_1)
            &= l_v (1 - q_1)
            \begin{aligned}[t]
            &[(1 - M_1) ((1 - m_2) (1 - p_1) + m_2 p_2)
            \\
            & + M_1 (m_1 (1 - p_1) + (1 - m_1) p_2)]
            \end{aligned}
        \\
        P(1 \rightarrow 2 \mid s = s_2)
            &= l_v q_2
            \begin{aligned}[t]
                &[M_2 ((1 - m_2) (1 - p_1) + m_2 p_2)
                \\
                &+ (1 - M_2) (m_1 (1 - p_1) + (1 - m_1) p_2)]
            \end{aligned}
        \\
        P(2 \rightarrow 1 \mid s = s_1)
            &= (1 - l_v) q_1
            \begin{aligned}[t]
                &[(1 - M_1) ((1 - m_2) p_1 + m_2 (1 - p_2))
                \\
                &+ M_1 (m_1 p_1 + (1 - m_1) (1 - p_2))]
            \end{aligned}
        \\
        P(2 \rightarrow 1 \mid s = s_2)
            &= (1 - l_v) (1 - q_2)
            \begin{aligned}[t]
                &[M_2 ((1 - m_2) p_1 + m_2 (1 - p_2))
                \\
                &+ (1 - M_2) (m_1 p_1 + (1 - m_1) (1 - p_2))]
            \end{aligned}
    \end{aligned}
\end{equation}

Working in mean field, one can write the following master equations:
\begin{equation}
    \label{eq:time_evol}
    \begin{aligned}
        % v and c depend on t
        \dv{p_1}{t} 
            &= (1 - p_1) P(2 \rightarrow 1 \mid s = s_1)
                - p_1 P(1 \rightarrow 2 \mid s = s_1)
        \\
        \dv{p_2}{t} 
            &= (1 - p_2) P(1 \rightarrow 2 \mid s = s_2)
                 - p_2 P(2 \rightarrow 1 \mid s = s_2)
    \end{aligned}
\end{equation}

% Note: the mobility matrix' Pearson r in this case is equal to $r_M = 1 - M_1 - M_2$.


\section*{Case of equal populations and mobility}
In the following, we will assume $N_{s_1} = N_{s_2}$, which implies $m_{k, j} =
\frac{M_{k, j}}{\sum_{k'} M_{k', j}}$. If we assume equal mobility, that is $M \equiv
M_1 = M_2$, we have $m_1 = m_2 = M$, and it follows that:
\begin{equation}
    \begin{aligned}
        P(1 \rightarrow 2 \mid s = s_1)
            &= l_v (1 - q_1)
            \left[
                2 M (1 - M) (p_1 + p_2 - 1) + 1 - p_1
            \right]
        \\
        P(1 \rightarrow 2 \mid s = s_2)
            &= l_v q_2
            \left[
                2 M (1 - M) (1 - p_1 - p_2) + p_2
            \right]
        \\
        P(2 \rightarrow 1 \mid s = s_1)
            &= (1 - l_v) q_1
            \left[
                2 M (1 - M) (1 - p_1 - p_2) + p_1
            \right]
        \\
        P(2 \rightarrow 1 \mid s = s_2)
            &= (1 - l_v) (1 - q_2)
            \left[
                2 M (1 - M) (p_1 + p_2 - 1) + 1 - p_2
            \right]
    \end{aligned}
\end{equation}
The result presented in \cref{eq:ses_ling_time_evol_eq_mob} then follows.

\end{document}
