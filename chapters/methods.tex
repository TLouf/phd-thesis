\documentclass[../thesis.tex]{subfiles}
\graphicspath{{\subfix{../figs/}}}
\begin{document}

\chapter{Methodology}
\label{ch:methods}

start with recap of previous chapters, into: ok so now what tools can we use. start with "old" way: classic linguistics, transition in ot computational and finish with complex systems stuff.
\cite{NguyenComputationalSociolinguistics2016}

different media to convey language
HERE written language exclusively, WHY: our methods, because that's what computers process better
SO we lose a lot of things that cannot be transcribed, or that are simply not because it would require considerable effort to do so  (accent, intonation), and access to spoken language is much more limited. also lose all non-verbal communication between human


Enormous amount of information exchanged through language. Data and metadata: . Someone's language tells a lot about them

methods with books "older data"

% data can say if model is wrong, but not if it's a good one: machine scientist stuff

\section{Data}

\subsection{What for}

\subsection{Traditional sources in linguistics}

\subsection{Twitter data}

\subsubsection{Accessing}

% \begin{lstlisting}[name=Test,language={Python}]
\begin{verbatim}
  {"str": "b"}
\end{verbatim}
  % print('lol')
% \end{lstlisting}


\subsubsection{Text processing}

\subsubsection{Infering geolocation}

\subsubsection{Caveats}
cover Twitter (biases but also basic technical details, API, what Tweet looks like), why remove HTs, blabla, language IDtion data driven analysis, cite Bruno papers eg

theoretical models: AS, MW 

finally computational methods, very general (data, PCA...)


\section{Models}

\subsection{What for}

\subsection{What kind}



\section{Source materials and tools}
Following the principles of open science, throughout my thesis, I have made all source
materials for my results openly accessible, whether they are codes\footnote{Hosted on
GitHub at \url{https://github.com/TLouf}} or datasets\footnote{Hosted on figshare at
\url{https://figshare.com/authors/Thomas_Louf/9441395}}, including this very
manuscript's\footnote{Hosted at \url{https://github.com/TLouf/phd-thesis}}. Equally importantly, I believe, I have strived to use almost exclusively free
and open source software in my work. I cannot realistically cite here all projects I
have relied on to carry out my work, but I can cite a few central ones. I wrote all my
code in the Python 3 programming language, using libraries such as
NumPy~\cite{HarrisArrayProgramming2020},
pandas~\cite{teamPandasdevPandas2020} or
GeoPandas~\cite{JordahlGeopandasGeopandas2020}. In their vast majority, figures
presented here were prepared with Matplotlib~\cite{HunterMatplotlib2D2007}, and
sometimes edited, or entirely drawn, with Inkscape\footnote{Available at
\url{https://inkscape.org}}.

This document was prepared using \LaTeX\ with the \texttt{classicthesis}
style\footnote{Hosted at \url{https://www.ctan.org/pkg/classicthesis}} developed by
Andr\'e Miede and Ivo Pletikosić, and the LaTeX Workshop extension\footnote{Hosted at
\url{https://github.com/James-Yu/LaTeX-Workshop}} of Visual Studio Code.


\section{Outline}



\end{document}
