\documentclass[../thesis.tex]{subfiles}
\graphicspath{{\subfix{../figs/}}}
\addbibresource{biblio.bib}
\begin{document}

\chapter{Language as a vector for communication}
\label{ch:lang_as_comm}

Language can generally be defined as how human beings structure their communications. As such, it is ubiquitous in any individual's life and in the workings of any human society.

It is so much so that researchers are unable to trace back to the origin of such a structured system of communication \cite{MullerLectureIX1861,StamInquiriesOrigin1976,GibsonOxfordHandbook2011,HauserMysteryLanguage2014}. Those who have dared do so have estimated language dates back tens or even hundreds of thousands of years \cite{NicholsOriginDispersal1998,ChomskyLanguageMind2004,BothaCradleLanguage2009,DediuAntiquityLanguage2013}. One fact is certain though: a huge diversity in language has emerged, as humans have come up with countless languages throughout history that evolved through contact with others. 

culture universal \cite{GreenbergLanguageUniversals2020,BrownDonaldHumanUniversals1991,}

Enormous amount of information exchanged through language. Data and metadata: . SOmeone's language tells a lot about them

Language is a dynamic object interacting with society necessarily. This is the starting premise of the whole field of sociolinguistics \cite{LabovSociolinguisticPatterns1973,TrudgillSociolinguisticsIntroduction2000,WardhaughIntroductionSociolinguistics2008}

\cite{LabovPrinciplesLinguistic1994,LabovPrinciplesLinguistic2001,LabovPrinciplesLinguistic2010}

competition between languages \cite{GrilloDominantLanguages1989,WardhaughLanguagesCompetition1987,BlommaertSociolinguisticsGlobalization2010}
de Saussure: "Among all the individuals that are linked together by speech, some sort of average will be set up: all will reproduce—not exactly of course, but approximately—the same signs united with the same concepts." \cite{deSaussureCourseGeneral2011}

For different speakers possess different quantities of 'linguistic capital' / language is socio-historical phenomenon / homogenous language community does not exist \cite{BourdieuLanguageSymbolic2009}

In this thesis, we investigate
- inter-language spatio-temporal evolution
- intra-language lexical variations in space, highlighting cultural differences
- intra-language spatial variations from the standard form, and its interplay with socio-economic status

different media to convey language
HERE written language exclusively, WHY: our methods, because that's what computers process better
SO we lose a lot of things that cannot be transcribed, or that are simply not because it would require considerable effort to do so  (accent, intonation), and access to spoken language is much more limited. also lose all non-verbal communication between human

things that push towards/against diversity (non exhaustive):
- need to comm = against
- culture: tricky one. extremely popular stuff, ("global culture" that nowadays can spread easily) can bring homogeneity (eg Hollywood). Other hand, this call for counter cultures, push against dominant one: "resist the invader" (catalan...). Group identity but at different levels, the closer the stronger
- ses: source of diversity. But tricky: this diversity actually brings by segregation

socio-linguistics / computational Nguyen

%% for other chapter
As mixing reduces chaos , great uniformisation happening at . Because major language shifts are bound to the passing of generations, this system has a considerable inertia. Despite this inertia intrinsic to language evolution, these changes are still taking place at dramatic speeds.

\end{document}
