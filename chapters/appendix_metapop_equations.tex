\documentclass[../thesis.tex]{subfiles}
\graphicspath{{\subfix{../figs/}}}
\begin{document}

\chapter[Approximate equations in a metapopulation]{Approximate evolution equations for our language competition model in a metapopulation framework}
\label{ch:appendix_metapop_equations}

Here we wish to derive a system of equations describing the evolution of the
metapopulation under reasonable assumptions. Let us first rewrite
\cref{eq:dt_bipref_model}, the global equations of our model, in terms of counts instead
of proportions:
\begin{equation}
\label{eq:bipref_cell_eq}
  \begin{aligned} 
    \dv{N_{A, i}}{t} &= \mu s N_{AB, i}
      \frac{
          \sum_{k}  (N_{A, k} + q N_{AB,k})
      }{
          \sum_{k} N_{k}
      }
    \\
    & \quad - c (1-\mu) (1-s) N_{A, i}
      \frac{
          \sum_{k} (N_{B,k} + (1-q) N_{AB, k})
      }{
          \sum_{k} N_{k}
      },
    \\[2ex]
    \dv{N_{B, i}}{t} &= \mu (1-s) N_{AB, i}
      \frac{
          \sum_{k} (N_{B, k} + (1-q) N_{AB,k})
      }{
          \sum_{k} N_{k}
      }
    \\
    & \quad - c (1-\mu) s N_{B, i}
      \frac{
          \sum_{k} (N_{A,k} + q N_{AB, k})
      }{
          \sum_{k} N_{k}
      },
  \end{aligned}
\end{equation}
for every cell $i$. Let us translate these to a metapopulation level, for which the
equations hold for the subpopulations of each cell $i$. We will follow what was done in
\cite{SattenspielStructuredEpidemic1995}, and divide every population $N_{L,i}$
according to their work destination $j$. We thus introduce the notation $N_{L,ij} (t)$
which is the number of $L$-speakers who are residents in $i$ and are at $j$ for work at
time $t$. It is such that
\begin{equation}
  N_{L,i} (t) = \sum_j N_{L, ij} (t).
\end{equation}
Then, the equations we want to solve to get the equilibrium points are, for every $i$,
\begin{equation}
\label{eq:fixed_points_eq}
  \begin{aligned} 
    \dv{N_{A, i}}{t} &\equiv \sum_j \dv{N_{A, ij}}{t} = 0 \\
    \dv{N_{B, i}}{t} &\equiv \sum_j \dv{N_{B, ij}}{t} = 0.
  \end{aligned}
\end{equation}

Regarding commuting, we will here use the notations from
\cite{BalcanModelingSpatial2010}, and introduce first $\sigma_{ij}$, the commuting rate
between the subpopulation $i$ and every other cell $j$. The return rate of commuting
individuals, that is the inverse of the timescale of their stay at work, is denoted
$\tau$. The subpopulation size evolution (summing over all languages) due to commuting
is then given by
\begin{equation}
\label{eq:subpop_commut_evo}
  \begin{aligned}
    \dv{N_{ii}}{t}
      &= \tau \sum_j N_{ij} (t) - \sum_j \sigma_{ij} N_{ii} (t)  \\
    \dv{N_{ij}}{t}
      &= \sigma_{ij} N_{ii} (t) - \tau N_{ij} (t).
  \end{aligned}
\end{equation}
Then, for monolinguals $A$, we can write the following:
\begin{equation}
  \begin{aligned}
    \dv{N_{A,ii}}{t}
      &= \text{A from every destination $j$ returning to their residence $i$} \\
      &- \text{A from $i$ leaving $i$ for work} \\
      &+ \text{AB from $i$ and currently at $i$ turning A} \\
      &- \text{A from $i$ and currently at $i$ turning AB},
  \end{aligned}
\end{equation}
which gives, using both the commuting part from \cref{eq:subpop_commut_evo} and the
language competition part from \cref{eq:bipref_cell_eq},
\begin{equation}
\label{eq:N_Aii_evo}
  \begin{aligned}
    \dv{N_{A,ii}}{t} 
      &= \tau \sum_j N_{A, ij} \\
      &- \sum_j \sigma_{ij} N_{A,ii} \\
      &+ \mu s (N_{ii} - N_{A, ii} - N_{B, ii}) \left( \frac{\sum_k (N_{A, ki} + q N_{AB, ki})}{\sum_k N_{ki}} \right)\\
      &- c (1-\mu) (1-s) N_{A, ii}  \left( \frac{\sum_k (N_{B, ki} + (1-q) N_{AB, ki})}{\sum_k N_{ki}} \right).
  \end{aligned}
\end{equation}
and similarly for every $j \neq i$:
\begin{equation}
  \begin{aligned}
    \dv{N_{A,ij}}{t}
      &= \text{A arriving at $j$ coming from their residence $i$} \\
      &- \text{A currently at $j$ returning to their residence $i$} \\
      &+ \text{AB from $i$ and currently at $j$ turning A} \\
      &- \text{A from $i$ and currently at $j$ turning AB},
  \end{aligned}
\end{equation}
which gives
\begin{equation}
\label{eq:N_Aij_evo}
  \begin{aligned}
    \dv{N_{A, ij}}{t} 
      &= \sigma_{ij} N_{A, ii} \\
      &- \tau N_{A, ij} \\
      &+ \mu s (N_{ij} - N_{A, ij} - N_{B, ij}) \left( \frac{\sum_k (N_{A, kj} + q N_{AB, kj})}{\sum_k N_{kj}} \right) \\
      &- c (1-\mu) (1-s) N_{A, ij} \left( \frac{\sum_k (N_{B, kj} + (1-q) N_{AB, kj})}{\sum_k N_{kj}} \right).
  \end{aligned}
\end{equation}

Now when we sum \cref{eq:N_Aij_evo} over $j$ and add \cref{eq:N_Aii_evo} to try to solve
for the system \cref{eq:fixed_points_eq}, we first see, as in \cite{SattenspielStructuredEpidemic1995}, that the commuting terms simplify. For the
language competition terms, there remains to estimate the $N_{A, ij}$. We will use
another result from \cite{BalcanModelingSpatial2010}, where they show that under the
assumption that $\forall ~ i, \tau \gg \sigma_i$, we can make the following approximation:
\begin{equation}
  \begin{aligned}
    N_{ii} &= \frac{N_i}{1 + \sigma_i / \tau} \\
    N_{ij} &= \frac{N_i \sigma_{ij} / \tau}{1 + \sigma_i / \tau}.
  \end{aligned}
\end{equation}
Let us introduce a matrix $\underline{\underline{\nu}}$ such that
\begin{equation}
  \begin{aligned}
    \forall ~ i,
      & ~ \nu_{ii} = \frac{1}{1 + \sigma_i / \tau}~,
    \\
    \forall ~ i, j \text{ such that } i \neq j,
      & ~ \nu_{ij} = \frac{\sigma_{ij} / \tau}{1 + \sigma_i / \tau}~,
  \end{aligned}
\end{equation}
so we can rewrite
\begin{equation}
  \forall ~ i, j, ~ N_{ij} = N_i \nu_{ij}.
\end{equation}
These counts, summed over all languages, are then constant. We can also use this
approximation for each language, by identification in the equation below:
\begin{equation}
  N_{A, ij} (t) + N_{B, ij} (t) + N_{AB, ij} (t) = (N_{A, i} (t) + N_{B, i} (t) + N_{AB, i} (t)) \nu_{ij}.
\end{equation}
We thus obtain the following equivalent equations under our assumptions:
\begin{equation}
  \begin{aligned}
    \dv{N_{A, i}}{t} 
        &= \mu s (N_i - N_{A, i} - N_{B, i})
        \sum_{j} \nu_{ij} [q (1 - \gamma_{B, j}) + (1 - q) \gamma_{A, j}]
    \\
        & \quad - c (1-\mu) (1-s) N_{A, i}
        \sum_{j} \nu_{ij} [(1 - q) (1 - \gamma_{A, j}) + q \gamma_{B, j}]
    \\
    \dv{N_{B, i}}{t} 
      &= \mu (1-s) (N_i - N_{A, i} - N_{B, i}) 
      \sum_{j} \nu_{ij} [(1 - q) (1 - \gamma_{A, j}) + q \gamma_{B, j}]
    \\
      & \quad - c (1-\mu) s N_{B, i}
      \sum_{j} \nu_{ij} [q (1 - \gamma_{B, j}) + (1 - q) \gamma_{A, j}],
  \end{aligned}
\end{equation}
where
\begin{equation}
  \begin{aligned}
  \gamma_{A, j} = \frac{
          \sum_{k} N_{A, k} \nu_{kj}
      }{
          \sum_{k} N_{k} \nu_{kj}
      },
  \\
  \gamma_{B, j} = \frac{
          \sum_{k} N_{B, k} \nu_{kj}
      }{
          \sum_{k} N_{k} \nu_{kj}
      }.
  \end{aligned}
\end{equation}

\end{document}
