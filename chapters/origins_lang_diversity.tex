\documentclass[../thesis.tex]{subfiles}
\graphicspath{{\subfix{../figs/}}}
\addbibresource{biblio.bib}
\begin{document}

\chapter{Where language diversity comes from}
\label{ch:origins_lang_diversity}

Language can generally be defined as
% a system that defines units carrying meaning (generally known as words) and how to combine those to convey more complex meaning.
a structured system that human beings use to communicate.
% how human beings structure their communications.
More specifically here, we use the term language to refer to natural languages,
meaning languages which have evolved naturally, or, said differently, that have not been
designed intentionally --- as opposed to programming languages, for instance. Our
objects of study are thus languages in the common sense of the word, that is coherent
systems that define words and how their combinations convey meaning --- like English,
Mandarin Chinese or Hindi, to cite the three most spoken nowadays. As a primary means of
communication, language
% such, it
is ubiquitous in any individual's life and in the workings of any human society. It is
so much so that it is considered a ``cultural universal'', meaning all known human
societies have some form of
language~\cite{GreenbergLanguageUniversals2020,BrownDonaldHumanUniversals1991,}. And it
is so much so that researchers are unable to trace back to the origin of such a
structured system of communication
\cite{MullerLectureIX1861,StamInquiriesOrigin1976,GibsonOxfordHandbook2011,HauserMysteryLanguage2014}.
Those who have ventured into this kind of inquiry have estimated that language dates
back tens or even hundreds of thousands of years
\cite{NicholsOriginDispersal1998,ChomskyLanguageMind2004,BothaCradleLanguage2009,DediuAntiquityLanguage2013}.
One fact is for certain though: for what could be colloquially called \emph{a very long
time}, human beings have come up with, innovated upon, used, and more generally
interacted with languages. It is then safe to say that human history must have seen a
huge diversity of languages emerge. What is more ambiguous, though, is how this
diversity is shaped through individuals' interactions, as they form societies. This is
the central question that defined the whole field of sociolinguistics
\cite{LabovSociolinguisticPatterns1973,TrudgillSociolinguisticsIntroduction2000,ChambersSociolinguisticTheory2007,WardhaughIntroductionSociolinguistics2008,LabovPrinciplesLinguistic2001},
which is also the broad subject of this thesis. In the following sections, we will touch
on the different roles of language in society that may bring about variation, or, on the
contrary, reduce existing linguistic diversity.

% \section{Language within the social context}
% this section with the following as subsections??
% % politics, prestige, economics
% % prestige

\section{Language as a vector for communication}
The first obvious function that language serves is to facilitate communication between
individuals, more specifically the kind of communication called \emph{verbal
communication}. To optimize language with regard to that function, there should only be
one single language, shared homogeneously among all individuals. This has not been the
case historically though, for many reasons, including historical and political ones, but
also a very down-to-earth one. It is the very simple fact that, for most of its history,
humanity has been spread around the Earth and unable to communicate at long distances.
There is one very well known example that illustrates this. Humans have been in America
for thousands of years: according to recently-found evidence, they have for more than
\SI{21000}{} years \cite{BennettEvidenceHumans2021}. Yet, the first lasting contact
between Europeans and indigenous Americans came less than 600 years ago. During all this
time, people on the two continents have had ample time to come up with new languages,
innovate upon existing ones, and mix within their own continent, at least partially.
Thus, on the scale of all these languages' histories, it is only very recently that the
two groups came into contact. Since then, things have accelerated extremely fast though.
First, transport has allowed long distance communication on the scale of months with
boats for roughly the past 500 years, and then on the scale of hours with planes since
the start of the last century. In the last two centuries, telecommunication has enabled
long distance and near-real-time communication, and it has truly been widely
democratized with the Internet in the last two decades. On the technical front, the
communication barriers between individuals across the globe seem to have come down. But
does this imply a push towards a reorganization of the world in what the philosopher
\citefirstlastauthor{McLuhanGutenbergGalaxy2008} called a \emph{global village}
\cite{McLuhanGutenbergGalaxy2008}? Does this imply a more interconnected world, and in
turn, that we will naturally tend towards the communication-optimal state of homogenous
language?

% mention Esperanto no more war whatever. language precedes divisions or other way around?

A physicist's intuition would say that the more individuals interact with one another,
the more language should \emph{thermalize}, or reach an equilibrium state of spatially
uniform and temporally constant language. Another view would be that, because it costs
energy for humanity to maintain language diversity, homogenization of language would be
both desirable and inevitable.
% But physics' success lies in its ability to provide good
% models of the world, that is, useful approximations of it. So while the model this
% intuition is drawn from was applied to the movement of physical bodies and has made the
% triumph of thermodynamics, is it here of any use, when trying to understand and make
% predictions about language diversity?
\citefirstlastauthor{deSaussureCourseGeneral2011}, a prominent linguist of the
late XIX$^\text{th}$ - early XX$^\text{th}$ century, seems to echo this view:

\begin{quote}
  Among all the individuals that are linked together by speech, some sort of average
  will be set up: all will reproduce --- not exactly of course, but approximately ---
  the same signs united with the same concepts. \cite{deSaussureCourseGeneral2011}
\end{quote}

% There is also some evidence supporting this idea.
% has faced considerable debate. First, the very hypothesis it relies on, that..., can be questioned. 

But this relies on the hypothesis that the global
society would tend toward complete interconnectedness. This idea was for instance
challenged by the anthropologist \citefirstlastauthor{DunbarNeocortexSize1992}, when he
suggested the existence of a maximum number of people one can maintain stable social
relationships with, which is known as Dunbar's number. Its existence was first
hypothesized~\cite{DunbarNeocortexSize1992,DunbarSocialBrain1998}, and later
demonstrated, not only on real-world social networks
\cite{HillSocialNetwork2003,McCartyComparingTwo2005}, but also for a massive, online one
\cite{GoncalvesModelingUsers2011}.
% TODO: make following better
On the other hand, while all individuals may not be completely interconnected, any two
individuals may be closer on the social network than one would expect. This is the claim
behind the famous idea of the six degrees of separation, or of the small world property
of social networks
\cite{deSolaPoolContactsInfluence1978,MilgramSmallWorld1967,TraversExperimentalStudy1977a,WattsCollectiveDynamics1998}.
This idea and the experiments behind it have received some criticism though
\cite{KleinfeldSmallWorld2002}. Also, more recent works focusing on online communication
networks have consistently found that distance still plays a major role in defining both
strong and weak ties
\cite{LeskovecPlanetaryscaleViews2008,TakhteyevGeographyTwitter2012a,Garcia-GavilanesTwitterAin2014}. 

The existence of a process of globalization is undisputed, but the idea that more
interconnectedness would create a global village has been challenged
\cite{NorrisCosmopolitanCommunications2009,BlommaertSociolinguisticsGlobalization2010}.
In the words of the sociolinguist
\citefirstlastauthor{BlommaertSociolinguisticsGlobalization2010}:

\begin{quote}
  The world has not become a village, but rather a tremendously complex web of villages,
  towns, neighbourhoods, settlements connected by material and symbolic ties in often
  unpredictable ways. That complexity needs to be examined and understood.
  \cite{BlommaertSociolinguisticsGlobalization2010}
\end{quote}
 
Nonetheless, there is some evidence of a homogenizing trend. English is on a steady path
to become a global language \cite{CrystalEnglishGlobal2010}. Most of the estimated
\SI{6000}{} languages that exist in the world today are endangered
\cite{CrystalLanguageDeath2000,GrenobleEndangeredLanguages1998,KraussWorldLanguages1992}
and getting replaced by a few dominant languages
\cite{GrilloDominantLanguages1989,WardhaughLanguagesCompetition1987}.
% It thus seems that studying language through the lens of communication alone 


% BUT apart from direct interactions, individual's language interacts with others in many different ways. first interventin to uniformise: political, nation states, like French state imposing French. second: cultural, mass media. third: economic: globalisation of economy. fourth 

One can then start to see the limitations of considering language as a neutral means of
communication. As important as the act of communicating itself are the reasons and
contexts of this communication.

% However, we have here implicitly only considered personal communication, but people can also be massively exposed to other languages through different media.


% \section{Language as a medium of power}
% this section with the following as subsections??

\section{Language and politics}
As \citefirstlastauthor{BourdieuLanguageSymbolic2009} argued, language is not only a
means of communication but also a medium of power \cite{BourdieuLanguageSymbolic2009}.
Some aspect of this idea has been popularized in the world of fiction with the concept
of the \emph{Newspeak} language in \citefirstlastauthor{Orwell19841950}'s
\citetitle{Orwell19841950} \cite{Orwell19841950}. It illustrates how a control over the
language spoken in society implies a better control over society itself. This has some
echo in the real world \cite{FowlerLanguageControl1979}, and not necessarily with
dystopian, \emph{Big Brother}, intentions of total control over individuals. When
nation-states were built, a common language was seen as a means to unite a nation
\cite{WrightCommunityCommunication2000}. This was particularly the case in
post-revolution France and imperial Great Britain
\cite{GrilloDominantLanguages1989,HigonnetPoliticsLinguistic1980}, where respectively
French and English were heavily pushed as the languages of higher status. Still today,
most constitutions specify one or a maybe a few languages as the languages of the state.
In theory, this would be beneficial as it enables the state to build a common ground to
guarantee equal opportunity, for instance with public education and the rule of law. It
also would not necessarily mean dropping local languages, but only learning a common
one, resulting then in a large population of multilinguals. In practice though, these
political pushes for a shared language have lead to the near extinction of many regional
languages and dialects. It is however not uncommon that a later reaction tried to
reverse this process, with policies switching roles completely. Politics can then
oppose homogenization and protect language diversity. Current examples include the
policies introduced to protect national languages against global English
\cite{SonntagLocalPolitics2003}, and also the ones to preserve regional languages and
dialects within nations \cite{KaplanLanguagePlanning1997}.
% All in all, whether
% politics push against or towards language diversity is simply a matter of the will of
% the people or governing elite.


% finally: interventions

\section{Language as a commodity} % on the market
Language can also be seen as part of the set of skills that an individual possesses and
may need to perform their job. Knowing a language has thus an economic value, and this
is particularly true in a globalized economy \cite{HellerCommodificationLanguage2010}.
Indeed, the world has not only become more interconnected in terms of communication, but
also in terms of trade. It is the aspect of globalization that has had the most impact
on our contemporary societies: in fact, when someone talks about globalization, most of
the time what they are referring to is economic globalization. In this context, good
command of a non-native language, like English in most cases, can often be a requirement
to apply for a job. As a result, the status of a language, or its perceived value in
society, can depend heavily on the value it is given by the market. As a result, as the
sociologist \citefirstlastauthor{BourdieuLanguageSymbolic2009} put it, individuals, as
they speak differently, possess different quantities of \emph{linguistic capital}
\cite{BourdieuLanguageSymbolic2009}.

Further, the manner with which one speaks a language can identify them as member of a
certain socio-economic group. Indeed, all languages have a number of varieties, some
with superior status. The standard variety of a language, when one is identified as such
by an official institution like a language academy, is often the most prestigious one. A
language has then a ``correct'' way to be spoken and written which is the one taught in
schools. As proven by the PISA reports of the OECD, its latest \cite{OECDWhereAll2019}
included, in many countries, linguistic proficiency of 15-year-olds strongly varies
based on their \ac{SES} of origin. Individuals from low socio-economic classes can then
be identified by their lack of command of the standard variety of their language, which
translates into a lesser linguistic capital. This can be detrimental to these parts of a
population, as these differences can entail segregation in several spheres of society,
notably on the job market, but also in social interactions.

% \section{Language as a social phenomenon}


% For different speakers possess different quantities of 'linguistic capital' / language is socio-historical phenomenon / homogenous language community does not exist / not only means of communication but also medium of power / different context people adapt their language / social structure appears in linguistic interactions \cite{BourdieuLanguageSymbolic2009}


\section{Language as a cultural trait}
As a vector for communication, language is also necessarily central in cultural
acquisition. It is even so intertwined with culture that some aspects of a culture may
be embedded directly in a language. It follows that the diffusion of a culture goes hand
in hand with the diffusion of a certain language. Here, language is to be understood in
the broad sense: it can either be a language like English that is diffused by Hollywood
cinema for instance, or a certain jargon within a language, like the (mostly English)
vocabulary associated to the Internet culture. As part of a culture, language may thus
contribute to building a sense of group identity to which individuals may adhere.
Conversely, rejecting the dominant, or mainstream, culture may also mean rejecting its
language, and protecting one's own. It may also mean coming up with one's own language,
as part of building a sub or counter-culture. This mechanism is not the only one at work
that pushes for language diversity. In \citeyear{UNESCOConventionSafeguarding2003}, the
\citeauthor{UNESCOConventionSafeguarding2003} adopted the
\citetitle{UNESCOConventionSafeguarding2003}, which states that language, ``as a vehicle
of the intangible cultural heritage'', is to be safeguarded against the effects of
globalization \cite{UNESCOConventionSafeguarding2003}. Also, very often, language
preservation policies are implemented based on the argument that cultural diversity
embedded in languages needs to be preserved
\cite{CrystalLanguageDeath2000,GrenobleEndangeredLanguages1998,KraussWorldLanguages1992}.

% it can also has the opposite effect of generating rejection. This is a general cultural process of  
% mass media and local identity 

% %% section goal of the thesis?? scope?

% In this thesis, we investigate
% - inter-language spatio-temporal evolution
% - intra-language lexical variations in space, highlighting cultural differences
% - intra-language spatial variations from the standard form, and its interplay with socio-economic status


% While sometimes
% considering languages as coherent, clearly-separated units (TODO ref chapter?) to study interlanguage interactions, we will also do away with this simplified view to study intra-language variations.


% Linguistics, and especially in its social branch, is thus at the interface of many entangled disciplines. We have shown how language and its study fall within the scope of various disciplines of social sciences, as we touched on subjects related to economics, communication science, human geography and political science.   

% \cite{LabovPrinciplesLinguistic1994,LabovPrinciplesLinguistic2001,LabovPrinciplesLinguistic2010}

\end{document}
