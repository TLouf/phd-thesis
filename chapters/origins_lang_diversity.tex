\documentclass[../thesis.tex]{subfiles}
\graphicspath{{\subfix{../figs/}}}
\addbibresource{biblio.bib}
\begin{document}

\chapter{What shapes diversity in language?} % in and of languages?
\label{ch:origins_lang_diversity}
 
\epigraph{
  [L]anguage [is] partly something originally given, partly that which develops
  freely. And just as the individual, however freely he may develop, can never reach the
  point at which he becomes absolutely independent, [...]
  % since true freedom on the contrary consists rather in freely appropriating that which is given, and consequently in being absolutely dependent through freedom,
  so too with language [...].
 }{
 \epigraphcite{KierkegaardJournals1936}
}

As a primary means of communication, language is ubiquitous in any individual's life and
in the workings of any human society. It is so much so that it is considered a
``cultural universal'', meaning all known human societies have some form of
language~\cite{GreenbergLanguageUniversals2020,BrownHumanUniversals1991}. And it
is so much so that researchers are unable to trace back to the origin of such a
structured system of communication
\cite{MullerLectureIX1861,StamInquiriesOrigin1976,GibsonOxfordHandbook2011,HauserMysteryLanguage2014}.
Those who have ventured into this kind of inquiry have estimated that language dates
back tens or even hundreds of thousands of years
\cite{NicholsOriginDispersal1998,ChomskyLanguageMind2004,BothaCradleLanguage2009,DediuAntiquityLanguage2013}.
One fact is for certain though: for what could be colloquially called \emph{a very long
time}, human beings have come up with, innovated upon, used, and more generally
interacted with languages. And this, all over the globe. Here, in particular, we will
focus on languages as understood in the common sense of the word, that is, coherent
systems that define linguistic signs and how their combinations convey meaning --- like
English, Mandarin Chinese or Hindi, to cite the three most spoken nowadays. Given how
long language has existed, and how many human beings have interacted with it, it is then
safe to say that human history must have seen a huge diversity in and of language
emerge. What is more ambiguous, though, is how this diversity is shaped through
individuals' interactions, as they form societies. This is the central question that
defines the whole field of sociolinguistics
\cite{LabovSociolinguisticPatterns1973,TrudgillSociolinguisticsIntroduction2000,ChambersSociolinguisticTheory2007,WardhaughIntroductionSociolinguistics2008,LabovPrinciplesLinguistic2001},
which is also the broad subject of this thesis. In the following sections, we will touch
on the different roles of language in society that may bring about heterogeneity, or, on
the contrary, push it towards homogeneity. When not explicitly specified, these effects
of heterogenisation or homogenisation will concern both language diversity --- that is,
differences between languages, when understood as coherent, clearly separated units --- and
language variation --- that is, differences of usage within a language.


\section{Language as a vector for communication}
The first obvious function that language serves is to facilitate communication between
individuals, and more specifically the kind of communication called \emph{verbal
communication}. To optimise language with regard to that function, there should be one
single language, shared homogeneously among all individuals. This has not been the case
historically though, for many reasons, including historical and political ones, but also
a very down-to-earth one. It is the very simple fact that, for most of its history,
humanity has been spread around the Earth and unable to communicate at long distances.
There is one very well known example that illustrates this. Humans have been in America
for thousands of years: according to recently-found evidence, they have for more than
\SI{21000}{} years \cite{BennettEvidenceHumans2021}. Yet, the first lasting contact
between Europeans and indigenous Americans came less than 600 years ago. During all this
time, people on the two continents have had ample time to come up with new languages,
innovate upon existing ones, and mix within their own continent, at least partially.
Thus, on the scale of all these languages' histories, it is only very recently that the
two groups came into contact. Since then, things have accelerated extremely fast though.
First, transport has allowed long distance communication on the scale of months with
boats for roughly the past 500 years, and then on the scale of hours with planes since
the start of the last century. In the last two centuries, telecommunication has enabled
long distance and near-real-time communication, and it has truly been widely
democratised with the Internet in the last two decades. On the technical front, the
communication barriers between individuals across the globe seem to have come down. But
does this imply a push towards a reorganisation of the world in what the philosopher
\citefirstlastauthor{McLuhanGutenbergGalaxy2008} called a \emph{global village}
\cite{McLuhanGutenbergGalaxy2008}? Does this imply a more interconnected world, and in
turn, that we will naturally tend towards the communication-optimal state featuring a
unique, homogenous language?


A physicist's intuition would say that the more individuals interact with each other,
the more language should \emph{thermalise}, or reach an equilibrium state of spatially
uniform and temporally constant language. Another view would be that, because it costs
energy for humanity to maintain language diversity, homogenisation of language would be
both desirable and inevitable.
% But physics' success lies in its ability to provide good
% models of the world, that is, useful approximations of it. So while the model this
% intuition is drawn from was applied to the movement of physical bodies and has made the
% triumph of thermodynamics, is it here of any use, when trying to understand and make
% predictions about language diversity?
\citefirstlastauthor{deSaussureCourseGeneral2011}, a prominent linguist of the
late XIX$^\text{th}$ - early XX$^\text{th}$ century, seems to echo this view:

\begin{quote}
  Among all the individuals that are linked together by speech, some sort of average
  will be set up: all will reproduce --- not exactly of course, but approximately ---
  the same signs united with the same concepts. \cite{deSaussureCourseGeneral2011}
\end{quote}

% There is also some evidence supporting this idea.
% has faced considerable debate. First, the very hypothesis it relies on, that..., can be questioned. 

But this relies on the hypothesis that the global
society would tend toward complete interconnectedness. This idea was for instance
challenged by the anthropologist \citefirstlastauthor{DunbarNeocortexSize1992}, when he
suggested the existence of a maximum number of people one can maintain stable social
relationships with, which is known as Dunbar's number. Its existence was first
hypothesised~\cite{DunbarNeocortexSize1992,DunbarSocialBrain1998}, and later
demonstrated, not only on real-world social networks
\cite{HillSocialNetwork2003,McCartyComparingTwo2005}, but also for a massive, online one
\cite{GoncalvesModelingUsers2011}.
On the other hand, while all individuals may not be completely interconnected, any two
individuals may be closer on the social network than one would expect. This is the claim
behind the famous idea of the six degrees of separation, or of the small world property
of social networks
\cite{deSolaPoolContactsInfluence1978,MilgramSmallWorld1967,TraversExperimentalStudy1977a,WattsCollectiveDynamics1998}.
This idea and the experiments behind it have received some criticism though
\cite{KleinfeldSmallWorld2002}. Also, more recent works focusing on online communication
networks have consistently found that distance still plays a major role in defining both
strong and weak ties
\cite{LeskovecPlanetaryscaleViews2008,TakhteyevGeographyTwitter2012a,Garcia-GavilanesTwitterAin2014}. 

In any case, the existence of a process of globalisation is undisputed, but the idea that more
interconnectedness would eventually lead to a global village has been challenged
\cite{NorrisCosmopolitanCommunications2009,BlommaertSociolinguisticsGlobalization2010}.
In the words of the sociolinguist
\citefirstlastauthor{BlommaertSociolinguisticsGlobalization2010}:

\begin{quote}
  The world has not become a village, but rather a tremendously complex web of villages,
  towns, neighbourhoods, settlements connected by material and symbolic ties in often
  unpredictable ways. That complexity needs to be examined and understood.
  \cite{BlommaertSociolinguisticsGlobalization2010}
\end{quote}
 
Networks of interpersonal communication are therefore complex systems that are hard to
summarise with a few simple properties. One can also start to see the limitations of
considering language as a neutral means of communication. In fact, as important as the
act of communicating itself are the reasons and contexts of this communication. It thus
seems that studying language through the lens of communication alone is too limiting to
gain a proper understanding of sociolinguistic phenomena.


% BUT apart from direct interactions, individual's language interacts with others in many different ways.

% However, we have here implicitly only considered personal communication, but people can also be massively exposed to other languages through different media.


% \section{Language as a medium of power}
% this section with the following as subsections??

\section{Language in the realm of politics}
As \citefirstlastauthor{BourdieuLanguageSymbolic2009} argued, language is not only a
means of communication, but also a medium of power \cite{BourdieuLanguageSymbolic2009}.
Some aspect of this idea has been popularised in the world of fiction with the concept
of the \emph{Newspeak} language in \citefirstlastauthor{Orwell19841950}'s
\citetitle{Orwell19841950} \cite{Orwell19841950}. It illustrates how a control over the
language spoken in society implies a better control over society itself. This has some
echo in the real world \cite{FowlerLanguageControl1979}, and not necessarily with
dystopian, \emph{Big Brother}-like intentions of total control over individuals. When
nation-states were built, pushing forward a common language was seen as a means to unite
a nation \cite{WrightCommunityCommunication2000}. This was particularly the case in
post-revolution France and imperial Great Britain
\cite{GrilloDominantLanguages1989,HigonnetPoliticsLinguistic1980}, where respectively
French and English were heavily pushed as the languages of higher status. Still today,
around the globe most countries' constitutions specify one or a maybe a few languages as
the languages of the state. In theory, this would be beneficial as it allows the state
to build a common ground to guarantee equal opportunity, for instance with public
education and the rule of law \cite{FergusonLanguageFactor1962}. It also would not
necessarily mean dropping local languages, but only learning a common one, resulting
then in a large population of multilinguals. In practice though, these political pushes
for a shared language have lead to the near extinction of many regional languages and
dialects: \SI{90}{\percent} of the languages that exist today may be replaced by a
handful of dominant languages over the course of this century, according to estimates of
the UNESCO \cite{UNESCOLanguageVitality2003}. Still, it is not uncommon that a later
reaction tried to reverse this process, with policies switching roles completely.
Politics can then oppose homogenisation and protect language diversity. Current examples
include the policies introduced to protect national languages against global English
\cite{SonntagLocalPolitics2003}, and the ones to preserve regional languages and
dialects within nations \cite{KaplanLanguagePlanning1997}. Politics thus very often push
for a given language ideology, which impacts whole languages
\cite{GalMultilingualism2006,RomaineBilingualMultilingual2012,FasoldSociolinguisticsSociety1984},
but also varieties within a language. Indeed, numerous languages have a rigorously
defined standard variety, sometimes defined by a state institution, like the
\textit{Académie Française} for French in France, or the \textit{Real Academia Española}
for Spanish in Spain, and often taught in all schools of a country. As such, it is
considered as the only ``correct'' way to speak and write the language. This tends to
uniformise language, as it favours the standard variety to the expense of all the others
\cite{MilroyIdeologyStandard2006,DavilaInevitabilityStandard2016}, usually called
non-standard or vernacular varieties, and which arise naturally within social groups.


\section{Language as a commodity}
\label{sec:lang_as_commodity}
Language can also be seen as part of the set of skills that an individual possesses and
may need to perform their job. Knowing a language therefore has an economic value, and
this is particularly true in a globalised economy
\cite{HellerCommodificationLanguage2010}. Indeed, the world has not only become more
interconnected in terms of communication, but also in terms of trade. It is the aspect
of globalisation that has had the most impact on our contemporary societies: in fact,
when someone talks about globalisation, most of the time what they are referring to is
economic globalisation. In this context, good command of a non-native language, like
English in most cases, can often be a requirement to apply for a job. As a result, the
status of a language, or its perceived value in society, can depend heavily on the value
it is given by the market. As the sociologist
\citefirstlastauthor{BourdieuLanguageSymbolic2009} put it, individuals, as they speak
differently, possess different quantities of \emph{linguistic capital}
\cite{BourdieuLanguageSymbolic2009}.

Further, the manner with which one speaks a language can identify them as member of a
certain socio-economic group Indeed, as we mentioned in the previous section, all
languages have a number of varieties, some with superior status, such as the standard
form. As proven by the PISA reports of the OECD, its latest included
\cite{OECDWhereAll2019}, in many countries, linguistic proficiency --- in the sense of
the standard language --- of 15-year-olds strongly varies based on their \ac{SES} of
origin. Individuals from low socio-economic classes can then be identified by their lack
of command of the standard variety of their language, which translates into a lesser
linguistic capital. This can be detrimental to these parts of a population, as these
differences can entail segregation in several spheres of society, notably on the job
market, but also in social interactions. 


\section{Language as a cultural trait}
\label{sec:lang_as_cultural_trait}
As a vector for communication, language is also necessarily central in cultural
acquisition. It is even so intertwined with culture that some aspects of a culture may
be embedded directly in a language. It follows that the diffusion of a culture goes hand
in hand with the diffusion of a certain language. Here, language is to be understood in
the broad sense: it can either be a language like English, that is diffused by Hollywood
cinema for instance, a certain jargon within a language, like the (mostly English)
vocabulary associated to the Internet culture, or any language variety. As part of a
culture, language may thus contribute to building a sense of group identity to which
individuals may adhere. Conversely, rejecting the dominant, or mainstream, culture may
also mean rejecting its language, and protecting one's own. It may also mean coming up
with one's own language, as part of building a sub or counter-culture. Indeed, different
social groups may have different language attitudes \cite{GarrettLanguageAttitudes2006},
meaning they may not value a language or a language variety the same way as other
groups. To go back to the opposition between standard and non-standard varieties,
different social groups may have different perceptions of what is the normal way to
speak \cite{KretzschmarLanguageVariation2010}. Some may thus oppose the language
ideology pushed by society as a whole or the state, which favours the standard form.
% It is thus not uncommon that some classes of the population prefer to use the non-standard
% form. This has been shown by prominent sociolinguists in the past, as they conducted
% field studies and analysed differences between socio-economic classes
% \cite{LabovSocialStratification1966,TrudgillSocialDifferentiation1974}.
This mechanism
is not the only one at work that pushes against homogenisation. In
\citeyear{UNESCOConventionSafeguarding2003}, the
\citeauthor{UNESCOConventionSafeguarding2003} adopted the
\citetitle{UNESCOConventionSafeguarding2003}, which states that language, ``as a vehicle
of the intangible cultural heritage'', is to be safeguarded against the effects of
globalisation \cite{UNESCOConventionSafeguarding2003}. Also, very often, language
preservation policies are implemented based on the argument that cultural diversity
embedded in languages needs to be preserved
\cite{CrystalLanguageDeath2000,GrenobleEndangeredLanguages1998,KraussWorldLanguages1992}.
Indeed, there is some evidence of a homogenising trend. English is on a steady path to
become a global language \cite{CrystalEnglishGlobal2010}. Most of the estimated
\SI{6000}{} languages that exist in the world today are endangered
\cite{CrystalLanguageDeath2000,GrenobleEndangeredLanguages1998,KraussWorldLanguages1992}
and getting replaced by a few dominant languages
\cite{GrilloDominantLanguages1989,WardhaughLanguagesCompetition1987}.


%% Language segregation sec? Lamanna paper has refs
% \section{Language as a defining block of social groups}
% for a mix of political, economic and cultural reasons, can bind people together, and at the same time separate them from others, as it distinguishes them from other groups. immigrants \cite{LamannaImmigrantCommunity2018} first obvious, but also SES \cite{abitbol}. Linguistic ghettos


\section{Scope and outline of the thesis}
If there was one chief takeaway from the last pages, it would be that language variation
is complex. There is variation between languages, but also within languages which all
have varieties, as there are as many varieties as speakers. There is variation in space
and time. And there is variation for many, often entangled, reasons. Linguistics, and
especially its social branch, is at the interface of many interwoven
disciplines. We have shown how language and its study fall within the scope of various
disciplines of social sciences, as we touched on subjects related to economics,
communication science, human geography and politics. Throughout this work, we will also
cross the boundaries between those, sometimes lying in-between. Our contribution is
humble: we will neither address every aspect of language variation,
% nor solve once and for all 
nor provide the definitive explanation for one aspect of the problem. We will rather
provide some further evidence and understanding of some of the phenomena at play.

Up next in \cref{ch:methods}, we will present the backbone of this work: the general
methodology that has been used throughout the thesis, along with the previous literature
that oriented our choices. This will be used to introduce concepts that permeate this
whole thesis along its two complementary streams of data analysis and theoretical
modelling.

After this thorough methodological review, in \cref{ch:multiling} we will investigate
interlanguage competition in space. There, we will consider languages as coherent units
that compete for speakers, which leads to geographically-embedded linguistic
communities. Our goal is first to observe these, second to measure the differences
between different kinds of language competition, and third to try to explain them.

In \cref{ch:ses_ling}, we will turn to intra-language variation, still with a
geographical component. We will see how socio-economic factors and social mixing can be
predictors of the variation between speakers of a single language. 

\Cref{ch:acr} deals again with intra-language variation in space, but this time
investigating how these can reveal different cultural values among individuals sharing
the same language. Cultural regions of the United States are thus inferred through the
analysis of social media posts.

Finally, \cref{ch:conclusion} will be the occasion for us to take a step back and reflect on the road
we have travelled during this thesis, and to envision what could be the next steps to go
forward in this general direction. 


\end{document}
