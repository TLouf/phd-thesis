\documentclass[../thesis.tex]{subfiles}
\graphicspath{{\subfix{../figs/}}}
\addbibresource{biblio.bib}
\begin{document}

\chapter{Where language diversity comes from}
\label{ch:origins_lang_diversity}

Language can generally be defined as
% a system that defines units carrying meaning (generally known as words) and how to combine those to convey more complex meaning.
a structured system that human beings use to communicate.
% how human beings structure their communications.
More specifically here, we use the term \emph{language} to refer to natural languages, meaning languages which have evolved naturally, or, said differently, that have not been designed intentionally --- as opposed to programming languages, for instance. Our objects of
study are thus languages in the common sense of the word, that is coherent
systems that define words and how their combinations convey meaning --- like English, Mandarin Chinese or Hindi, to cite the three most spoken nowadays.
As
a primary means of communication, language
% such, it
is ubiquitous in any individual's life and in the workings of any human
society. It is so much so that it is considered a ``cultural universal'', meaning all
known human societies have some form of
language~\cite{GreenbergLanguageUniversals2020,BrownDonaldHumanUniversals1991,}.
And it
is so much so that researchers are unable to trace back to the origin of such a
structured system of communication
\cite{MullerLectureIX1861,StamInquiriesOrigin1976,GibsonOxfordHandbook2011,HauserMysteryLanguage2014}.
Those who have ventured into this kind of inquiry have estimated that language dates back tens or even hundreds of
thousands of years
\cite{NicholsOriginDispersal1998,ChomskyLanguageMind2004,BothaCradleLanguage2009,DediuAntiquityLanguage2013}.
One fact is for certain though: for what could be colloquially called \emph{a very long
time}, human beings have come up with, innovated upon, used, and more generally
interacted with languages. It is then safe to say that human history must have seen a huge
diversity of languages emerge. What is more ambiguous, though, is how this diversity is
shaped through individuals' interactions, as they form societies. This is the defining question of the whole field of sociolinguistics \cite{LabovSociolinguisticPatterns1973,TrudgillSociolinguisticsIntroduction2000,ChambersSociolinguisticTheory2007,WardhaughIntroductionSociolinguistics2008,LabovPrinciplesLinguistic2001}, that is also the core question that we address in this thesis. In the following sections, we will
touch on the different roles of language in society that may bring about variation, or, on the contrary, reduce existing linguistic diversity.

% \section{Language within the social context}
% this section with the following as subsections??
% % politics, prestige, economics
% % prestige

\section{Language as a vector for communication}
The first obvious function that language serves is to facilitate communication between
individuals, more specifically the kind of communication called \emph{verbal
communication}. To optimize language with regard to that function, there should be only
one homogeneously shared among all individuals. This has not been the case historically
though, for many reasons, including historical and political ones, but also a very
down-to-earth one. It is the very simple fact that, for most of its history, humanity
has been spread around the Earth and unable to communicate at long distances. There is
one very well known example that illustrates this. Humans have been in America for
thousands of years: according to recently-found evidence, they have for more than
\SI{21000}{} years \cite{BennettEvidenceHumans2021}. Yet, the first lasting contact
between Europeans and indigenous Americans came less than 600 years ago. During all this
time, people on the two continents have had ample time to come up with new languages,
innovate upon existing ones, and mix, at least partially. Thus, on the scale of all
these languages' histories, it is only very recently that the two groups came into
contact. Since then, things have accelerated extremely fast though. First, transport has
allowed long distance communication on the scale of months with boats for roughly the
past 500 years, and then on the scale of hours with planes since the start of the last
century. Then, telecommunication has enabled long distance and near-real-time
communication in the last two centuries, and it has truly been widely democratized with
the internet in the last two decades. Are the new forms of communication brought by
these technological shifts then pushing a reorganization of the world towards McLuhan's
\emph{global village} \cite{McLuhanGutenbergGalaxy2008}? Does this imply that we will
naturally tend towards the communication-optimal state of homogenous language?

% mention Esperanto no more war whatever. language precedes divisions or other way around?

A physicist's intuition would say that the more individuals interact with one another,
the more language should \emph{thermalize}, or reach an equilibrium state of maximum
entropy. Another view would be that, because it costs energy for humanity to maintain
language diversity, homogenization of language would be both desirable and inevitable.
But physics' success lies in its ability to provide good models of the world, that is,
useful approximations of it. So while the model this intuition is drawn from was applied
to the movement of physical bodies and has made the triumph of thermodynamics, is it
here of any use, when trying to understand and make predictions about language
diversity? Ferdinand de Saussure, a prominent linguist of the late XIX$^\text{th}$-early
XX$^\text{th}$ century, seems to echo this view:

\begin{quote}
  Among all the individuals that are linked together by speech, some sort of average
  will be set up: all will reproduce --- not exactly of course, but approximately ---
  the same signs united with the same concepts. \cite{deSaussureCourseGeneral2011}
\end{quote}

% There is also some evidence supporting this idea.

But this view has also been questioned, starting with its central hypothesis that the global society would tend toward complete interconnectedness.
This idea was for instance challenged by the anthropologist Robin Dunbar, when he suggested the existence of a maximum number of people one can maintain stable social relationships with, known as Dunbar's number. Its existence was first hypothesized~\cite{DunbarNeocortexSize1992,DunbarSocialBrain1998}, and later demonstrated, not only on real-world social networks \cite{HillSocialNetwork2003,McCartyComparingTwo2005}, but also for a massive, online one \cite{GoncalvesModelingUsers2011a}.
The existence of a global village is then arguable

\begin{quote}
  The world has not become a village, but rather a tremendously complex web of villages,
  towns, neighbourhoods, settlements connected by material and symbolic ties in often
  unpredictable ways. That complexity needs to be examined and understood.
  \cite{BlommaertSociolinguisticsGlobalization2010}
\end{quote}
 
small world, but that is contested, but anyway no death of distance: 
. The % The social network of individuals may have changed slightly, 

Nonetheless, there is some evidence of a homogenising trend. English is on a steady path to become a global language \cite{CrystalEnglishGlobal2010}. Most of the estimated \SI{6000}{} languages that exist in the world today are endangered \cite{CrystalLanguageDeath2000,GrenobleEndangeredLanguages1998,KraussWorldLanguages1992} and getting replaced by a few dominant languages \cite{GrilloDominantLanguages1989,WardhaughLanguagesCompetition1987}.


% BUT apart from direct interactions, individual's language interacts with others in many different ways. first interventin to uniformise: political, nation states, like French state imposing French. second: cultural, mass media. third: economic: globalisation of economy. fourth 

One can then start to see the limitations of considering language as a neutral means of communication. As important as the act of communicating itself are the reasons and contexts of this communication. As Pierre Bourdieu argued, language is not only a means of communication but also a medium of power \cite{BourdieuLanguageSymbolic2009}. 

% However, we have here implicitly only considered personal communication, but people can also be massively exposed to other languages through different media.


% \section{Language as a medium of power}
% this section with the following as subsections??

\section{Language as a political weapon}
As Pierre Bourdieu argued, language is not only a means of communication but also a medium of power \cite{BourdieuLanguageSymbolic2009}. 
% However, this process has been far a completely natural one. Often, people who believed in the reciprocal idea, that to link individuals together, a common language should unite them, pushed for homogenisation. This was extensively done by leaders of nation-states. Many constitutions  specify a single language as the language of the state.. At the global level, Esperanto
% example above also has not been entirely natural, emerging from interpersonal communication: lcoal politics of global english 
% finally: interventions

\section{Language on the market} % as a commodity
For different speakers possess different quantities of 'linguistic capital' \cite{BourdieuLanguageSymbolic2009}
% \cite{BourdieuLanguageSymbolic2009,BlommaertSociolinguisticsGlobalization2010}
% \cite{BlommaertSociolinguisticsGlobalization2010}
% communication can also be with oneself,, personal, internal language


% \section{Language as a social phenomenon}



% For different speakers possess different quantities of 'linguistic capital' / language is socio-historical phenomenon / homogenous language community does not exist / not only means of communication but also medium of power / different context people adapt their language / social structure appears in linguistic interactions \cite{BourdieuLanguageSymbolic2009}


\section{Language as a cultural trait}



In this thesis, we investigate
- inter-language spatio-temporal evolution
- intra-language lexical variations in space, highlighting cultural differences
- intra-language spatial variations from the standard form, and its interplay with socio-economic status



things that push towards/against diversity (non exhaustive):
- need to comm = against
- culture: tricky one. extremely popular stuff, ("global culture" that nowadays can spread easily) can bring homogeneity (eg Hollywood). Other hand, this call for counter cultures, push against dominant one. Group identity but at different levels, the closer the stronger
- ses: source of diversity. But tricky: this diversity actually brings by segregation


While sometimes
considering languages as coherent, clearly-separated units (TODO ref chapter?) to study interlanguage interactions, we will also do away with this simplified view to study intra-language variations.
%% section goal of the thesis?? scope?

Linguistics, and especially in its social branch, is thus at the interface of many entangled disciplines. We have shown how language and its study fall within the scope of various disciplines of social sciences, as we touched on subjects related to economics, communication science, human geography and political science.   

\cite{LabovPrinciplesLinguistic1994,LabovPrinciplesLinguistic2001,LabovPrinciplesLinguistic2010}

\end{document}
